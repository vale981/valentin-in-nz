\chapdate{08.11.2016}
\chapter{Regen}

Hallo mal wieder. Wie es aussieht, gibt es mal wieder einen neuen
Blogpost. Es scheint paradox, ist jedoch zumindest für mich zutreffend:
Partout habe ich das Gefühl, etwas nicht erwähnt zu haben. Eigentlich
sollte man sich doch an seine letzte Woche erinnern können\ldots{} Nun,
wie auch immer. Das, was ich aus dem konfusen Brei, der sich bei mir
Gedächtnis schimpft, fischen kann und dann auch noch ausreichend
interessant erscheint, folgt nun.

Die letzte Woche war vor allem durch ein ausgezeichnetes Regenwetter
gekennzeichnet. Nicht einfach nur Regen, es kommt noch besser, auch viel
Grau und noch mehr Sauna (Sonne erhitzt zwischen Boden und Wolken
festsitzende Luft). Somit hatte ich wenig im Garten, dafür aber mehr im
Haushalt zu tun und kann nun sehr effizient den gesamten Kern in einem
Stück aus einer Walnuss schälen. Ja und die Tage verflogen. Ich weiß
wirklich nicht wie, aber am Ende des Tages hatte ich immer keine Zeit
mehr :). Einen Abend haben wir "Catch me if You Can" gesehen, trés
amusant, wenigstens an das kann ich mich noch erinnern.

Mir kam die zündende Idee ein Arbeitstagebuch zu führen, sodass ich
wenigstens mit meinen Großtaten prahlen kann. Bisher sieht das Ganze
recht ambitioniert so aus: (wobei ich schon einmal interpolieren
musste\ldots)

\begin{description}
\item[1. Nov] cutting flax, bundling it, digging it out
\item[2. Nov] Cleaning Lamps in ceiling, cleaning inside of the car 100%
\item[3. Nov] Filling the flax hole, cleaning plant storage, salt-watering weeds
\item[4. Nov] Cracking Walnuts, Weeding and Pruning in Community Gardens
\item[5. Nov] Free Day, Hiking
\item[6. Nov] Nut Shelling, Cleanup of Garden Space, Sorting Pots
\item[7. Nov] Vacuuming, Free Day (Museum)
\item[8. Nov] Pruning, Weeding (long, 5h+)
\end{description}

Am Samstag dann hatte ich einen freien Tag und entfloh in den relativ
regenfreien Süden auf eine Wanderung am Kapiti Coast. Nun, das Ganze ist
ausreichend photographisch dokumentiert und somit habe ich nur zu
berichten, dass ich auf dem Rückweg fast im aufgewühlten Meer baden
wollte, mich aber nicht dazu durchringen konnte. Ich habe dann aber mit
dem Auto ein paar Runden gedreht \ldots{} nicht ins Meer :P, aber durch
die Umgebung. Paraparaumu ist doch ein ganzes Stück größer, als ich
zunächst annahm.
\begin{figure}[h]
  \centering
  \includegraphics[width=\textwidth]{11/kapiti.JPG}
  \caption*{Kapiti Coast mit Kapiti Island}
\end{figure}

Gestern dann unternahm ich einen noch besser durch Photographie
dokumentierten Ausflug in das "Southward Car Museum". Sir Len Southward
fing irgendwann im letzten Jahrhundert an, als Mechaniker eine
Automobilwerkstatt zu führen. Das verhalf ihm dann zu einem Reichtum,
der nur durch das Sammeln älterer, neuerer, schöner, hässlicher,
ausgefallener, \ldots{} Automobile umgesetzt werden konnte. Und heute
können wir dank seiner Generosität das Ganze als Museum erleben. Allein
mit 10\% der Ausstellung verbrachte ich meine erste Stunde und las fast
alle kleinen Täfelchen zu den Exponaten. Später dann sparte ich mir das,
um zugunsten der vollständigen Besichtigung des Museums (die Halle, ein
Motorradbalkon und ein großer Keller) ein schnelleres Tempo an den Tag
zu legen, nur noch interessantere Exponate näher zu studieren und meinen
Aufenthalt von weiteren 9 Stunden auf erträgliche 3 (insgesammt also 4)
Stunden zu beschränken.

\begin{figure}[h]
  \centering
  \includegraphics[width=\textwidth]{11/trabi.JPG}
  \caption*{Trabi im Southward Car Museum}
\end{figure}

Besonders interessant waren bei all dem die alten Kuriositäten, wie das
erstaunlicherweise zu seiner Zeit (in den 30iger Jahren des 20. Jhd.)
recht populäre Phänomobil. Das Phänomobil ist eine Art
Dreiradswägelchen, bei dem der Motor direkt über dem Vorderrad sitzt und
sich beim Steuern mitdreht. Man lenkt dabei mit einer rechtwinklig zur
Lenkachse angebrachten Stange und steuert den mit zwei roten Propellern
gekühlten Motor über zwei Ventile.
\begin{figure}[h]
  \centering
  \includegraphics[width=\textwidth]{11/phaeno.JPG}
  \caption*{Ph\"anomobil}
\end{figure}

Desweiteren fand ich viel Freude an diversen Sportwagen, aber auch an
einem frühen Mercedes mit Flugzeugmotor und wassergekühlten
Bremsen. Man konnte den Dreitonner nur im dritten Gang fahren, da bei
den ersten beiden nur ein Burnout (Reifendurchdrehen) zu erwarten
war. Neben allerhand verrückter Custom-Cars gab es auch verrückte Mini
Autos wie die BMW Isetta (hergestellt nach einer Linzens einer
italienischen Firma mit einigen Verbesserungen seitens BMW), Oldtimer,
motorisierte Tandemfahrräder und Flugzeuge. Ein höchst interessanter
Aufenthalt, besonders, wenn man sich die Produktionszahlen einiger
Modelle ansieht. Wenn die alle heute noch fahren würden\ldots{} Auch
der Leistungsanstieg von mickrigen 8-12 PS der motorisierten
("Horseless" fancy, fancy!!) Pferdekutschen zu Sportwagen mit 300 PS
und mehr. Zudem gab es zu Anfang einen recht großen Markt für die
einfach zu handhabenden und leisen Elektromobile, die dann aber von
der Entwicklung des Verbrennungsmotors überholt wurden. Heute noch
wird die Sammlung stetig erweitert und erstaunliches an halb
verwrackten Wagen geleistet.

Heute habe ich zur Abwechslung mal schönes Wetter und hart gearbeitet.
Dabei hat mir Grübelei und Gudruns Modellansatz Podcast die Zeit
versüßt. Schon wieder ein neues Wunschstudienfach: Technomathematik!
Fast wie Kybernetik, aber noch vielseitiger.

Nun denne, jetzt gehts für den Sonnenuntergang auf zum Strand! Bis zum
nächsten Mal.
\begin{figure}[h]
  \centering
  \includegraphics[width=\textwidth]{11/sundown.JPG}
\end{figure}
