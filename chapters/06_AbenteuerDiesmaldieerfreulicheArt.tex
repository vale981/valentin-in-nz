\chapdate{19.09.2016}
\chapter{Abenteuer (Diesmal die erfreuliche Art)}

Es war eine schöne Woche. Viel Regen, aber auch viele interessante
Gespräche. Man reist durch Neuseeland und trifft: Deutsche. Wir waren 4
Deutsche und ein Österreicher.

Paora, unser WWOOFing Host, lud uns Anfang der Woche zu einem ominösen
'Cultural Event' ein. Im Laufe der Woche konkretisierte sich der

\begin{enumerate}
\tightlist
\item
  Geburtstag seines Neffen als dieses Event heraus. Ein
\end{enumerate}

merkwürdiger Gedanke als Wildfremder auf einen Geburtstag zu gehen,
eingeladen vom Onkel des 'Geburtstagskindes'. Wir wurden recht bald
dahingehend beruhigt, dass es ein sehr formelles Fest mit vielen Reden
und ähnlichem sei (was meine Zweifel aber nicht völlig ausräumte).

Geweckt vom Gesang des Mobiltelefons von Micha (einem WWOOFer) brachen
wir sechs Uhr in der Frühe auf, um 8 Uhr irgendwo im Nirgendwo bei einem
Maori Marai, gedacht für Feierlichkeiten, anzukommen. Da es zu diesem
Zeitpunkt schon nichts mehr zu tun gab, ging es weiter die Straße (den
Feldweg) hinab, um dabei zu helfen, frisch unter der Erde gebackene
Fleischpacken in handliche Stücke zu zerlegen. Ich habe noch nie im
Leben solch eine Fettschicht von einem Tisch wischen dürfen. Danach
schloss sich der offizielle Teil des Geburtstages an.

Der einundzwanzigste Geburtstag markiert bei den Maori den Eintritt in
das Erwachsenenalter und ist damit fast noch wichtiger als unser
achtzehnter Geburtstag. Wo bei uns jeder Geburtstag anders, mehr oder
weniger informell ist, so greift bei den Maori die Tradition, die
bewundernswert bewahrt wird und, wie man uns verriet, in viele
Festivitäten gipfelt. So traten wir Gäste durch das (symbolische)
Haupttor, begleitet vom Sprechgesang der Familienältesten, einer
beeindruckenden Frau mit schwarz tätowierten Lippen, in den Marai ein,
die Frauen zuerst und danach die Männer. Danach folgten Wechselreden von
Gastgeber und Gästen. Zum einen um den 21 -jährigen in die Welt der
Erwachsenen einzuführen, aber auch um die guten Absichten als Besucher
zu erklären und von den Gastebern akzeptiert zu werden. Anschließend gab
es ein großes Essen, unterbrochen von zahlreichen (und langen) Reden und
beeindruckenden und lautstarken Einlagen seitens der Jungen Männer.
Schlussendlich halfen wir WWOOFer die Tische abzuräumen (schon das
zweite Extrem an diesem Tag: ich habe noch nie so viel Chaos auf einem
Tisch gesehen :P) und das Geschirr zu spülen. Danach ging es ans Kuchen-
bzw. Muffinbuffet und den unformellen Teil. Wir hatten viele
interessante Gespräche mit den Gästen, die uns so herzlich und
selbstverständlich als Ihresgleichen betrachteten, wie es in Deutschland
wohl nicht möglich gewesen wäre. Ein unvergleichliches und unbezahlbares
Erlebnis, kaum wieder gut zu machen, selbst nicht durch Küchenarbeit :).

Nun bin ich weiter gezogen: Nach Levin an der Ost- (für unsere Begriffe
West-) Küste zu einer älteren Dame, um im Garten zu helfen. Eine
wunderbare und herzliche Frau, bei der man sich sofort Zuhause fühlt.
Sie selbst lernt gerade Ukulele (im buchstäblichen Sinne. Ich höre es
gerade durch die Tür schallen :).). Ihr Sohn macht Musik für Kinder
(bzw. ist Instrumentallehrer). Nun sehen wir mal was die Woche bringt.
