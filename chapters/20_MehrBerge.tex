\chapdate{05.04.2017}
\chapter{Mehr Berge}
\begin{figure}[h]
  \centering
  \includegraphics[width=\textwidth]{20/alpineview.JPG}
  \mycap{Eine der vielen Gebirgsaussichten}
\end{figure}

Welch Turbulenzen! Eddies und Wirbel haben den Blog ganz aus meinem
Geiste geblasen! Jetzt ist es aber höchste Eisenbahn, die nächste
Fortsetzung zu schreiben. Also dann mal ran an den Speck.

Wo waren wir stehen geblieben \ldots{}

Endlich einmal hatte ich das Vergügen, meinen Schlafsack auch einmal
sinnvoll zu nutzen und wirklich, das Geld hat sich gelohnt und ich kam
warm durch eine recht kühle Nacht.

Am nächsten Morgen war ich einmal mehr dabei, meine sieben (acht)
Sachen zusammenzusuchen und weiter zu fahren, als, es kommt uns
bekannt vor, ich mit einem netten, deutschen Informatiker ins Gespräch
kam. Wir hatten schon am vorherigen Abend zusammen eine kleine
Exkursion zu ein paar beleuchteten Wasserfällen unternommen und
wollten jetzt eine kleine drei-Stunden-Wanderung auf dem Arthur's Pass
Walkway angehen.

\begin{figure}[h]
  \centering
  \includegraphics[width=\textwidth]{20/rubikon.JPG}
  \mycap{Unser Rubikon}
\end{figure}
Frisch und munter ächzten wir dahin, als wir, empört über unserer
beider Kondition, ein paar Stufen zu einem weiteren Wasserfall
emporkletterten. Der Wasserfälle gibt es viele in Neuseeland, fast zu
viele, als dass man sie würdigen könnte, aber an Größe konnte bisher
keiner mit dem vor uns dahin rauschenden Exemplar mithalten! Weiter
ging es mit allerhand Abstechern, bis wir zu einer kleinen Brücke
gelangten, nach der der Weg nur noch für ``Mountaineers''
(Bergsteiger) geeignet war. Und während all dem gab es eine so
wunderbare Szenerie. Jeder Berg hat seine Eigenheiten, mitunter sogar
eine andere Vegetation und geht man nur ein paar Minuten voran, hat
man wieder eine völlig andere Perspektive und kann sich auf ein Neues
sattsehen. Auf dem Rückweg quälte ich mich ein bisschen, da ich in der
Erwartung, nur sechs Kilometer zu laufen, keine Verpflegung
mitgenommen hatte! Zurück im Hostel stürzte ich mich nach dieser 16km
Wanderung auf meinen Vorrat an Käse und Supermarkt-Baguette.  Alles
schmeckt delicieuse, wenn man nur genügend Hunger hat.

Nach einer kleinen Ruhepause ging es ab nach Christchurch. Ich hatte
Schwierigkeiten meine Konzentration auf die Straße bei solch einer
Szenerie aufrecht zu erhalten. Ein paar Anblicke mit kahlen Bergen, die
wie gigantische Sand- und Schutthaufen aussahen, erinnerten sogar an
Ronneburg vor der Bundes-Gartenschau :P.

\begin{figure}[h]
  \centering
  \includegraphics[width=\textwidth]{20/castle.JPG}
  \mycap{Steinformationen am Castle Rock}
\end{figure}
Auf dem Wege wollte ich mir noch den berühmten Castle-Rock mit seinen
Steinformationen ansehen und folgte brav dem Navi, dass mich dann aber
in ein Feriendorf ohne erkennbaren Zugang zum Hügel lotste. Enttäuscht
fuhr ich vondannen, nur um fünf Minuten später und voller Freude den
richtigen Parkplatz zu entdecken. Der Farmer, der das umliegende Land
sein Eigen nennt, hatte nicht an Warnschildern und Draht gespart, sodass
man sich fragte, ob er nun Touristen oder Rinder einzäunt.

\begin{figure}[h]
  \centering
  \includegraphics[width=\textwidth]{20/spire.JPG}
  \mycap{Ein Gottesanbeter}
\end{figure}
Der Castle-Rock selbst sieht aus, wie eine Cyberpunk Steinstadt oder das
Gebiss eines Riesens und konnte mich, selbst nach all dem in Neuseeland
Gesehenen, noch erfreulich überraschen. Reichlich beeindruckt von meinem
Tag legte ich auch die letzten Kilometer nach Christchurch zurück.

Das Hostel, in dem ich die Nacht verbrachte, kann ich wohl getrost zu
meinen Favoriten zählen. Klein aber fein und sehr gemütlich. So
freundete ich mich auch gleich mit einer sympathischen Amerikanerin an
und wir erzählten so über dies und jenes. Ihr Rückflug nach Amerika ging
über ein von Trumps Travel-Ban betroffenes Land, dessen Konsulat
freundlicherweise ein Schreiben an ihre Airline verfasste, da diese
ihren Flug nicht umbuchen wollte.

Ein Tag mit noch größeren Erlebnissen als der letzte!

Danke für's mitmachen! Schalten Sie auch morgen wieder ein, denn es
folgt: Christchurch.
