\chapdate{09.12.2016}
\chapter{Er lebt}

Holla. Er lebt noch. Nach ein paar interessanten, angespannten und
entspannten Wochen nehme mir endlich einmal Zeit, einen überfälligen und
hoffentlich heiß ersehnten Blogeintrag zu schreiben.

Selten kam mir ein dermaßen praktischer Geistesblitz zu passe. So
einleuchtend im Charakter die Idee auch war, lange blieb Sie mir dennoch
verborgen. Die Rede ist von einem von mir nunmehr täglich in strenger
Disziplin geführten persönlichen Tagebuch als kleines accompagnement zu
meinem Work-Diary. Endlich muss man sich keine Sorgen machen, das
Unvergessliche, Erlebte in seiner schieren Masse zu vergessen. Jeden
Abend tippe ich also mehr oder manchmal auch, der Müdigkeit geschuldet,
weniger einen kurzen Tagesbericht inklusive neuer Erkenntnisse und
zwangsläufig auch Fragen. Zu eurem Leidwesen resultierte das Ganze in
einer BlogPostFaulheit, der ich mit diesem Eintrag ein Ende zu setzen
versuche.

Hmm. Wo waren wir stehen geblieben \ldots{} Ich war zurück von Fiji und
bin nun\ldots{}

\ldots{} in Whakatane, genauer: nahe Thornton Beach. Die Reise von
Wellington habe ich in zwei Hälften geteilt, um die Fahrt auch genießen
zu können. Gesehen habe ich den Tongariro National Park, oder besser:
ich habe ihn auf State Highway One durchfahren (wärmstens zu empfehlen),
ich bin auch gewandert und im eisigen Lake Taupo geschwommen.
\begin{figure}[h]
  \centering
  \includegraphics[width=\textwidth]{13/lake_taupo.JPG}
  \mycap{Blick auf Lake Taupo}
\end{figure}
\begin{figure}[h]
  \centering
  \includegraphics[width=\textwidth]{13/hw1.JPG}
  \mycap{State Highway One}
\end{figure}

Übernachtet habe ich in einem Backpacker Hostel und war sehr angenehm
überrascht. Sauber, leise, gemütlich und preislich sehr attraktiv stand
die Unterkunft, wie ich nun weiß, in angenehmen Kontrast zu anderen
Herbergen. Am zweiten Tag verfuhr ich mich erst einmal gründlich und
endete an einer abgesperrten Forrest-Road, dann an einer Weiteren und
schließlich auf dem Highway.

Pünktlich zum Lunch fand ich beim dritten
Versuch das Haus meiner Hosts und siehe da, eine weitere sehr positive
Überraschung wartet auf mich. Wirklich direkt in den Sanddünen gelegen
und liebevoll gestaltet, ein Ort, besser als jedes Ferienhaus. Es lebt
sich sehr schön bei den Niederländern Wilhelmina und Gerrit und
besonders das Essen ist unübertrefflich. Die Beiden haben eine
unglaubliche Menge an Olivenbäumen und stellen mit ersten Plätzen und
Goldzertifikaten ausgezeichnetes Oliven-Öl her (und das erst seit
wenigen Jahren!).
\begin{figure}[h]
  \centering
  \includegraphics[width=\textwidth]{13/view_terace.JPG}
  \mycap{Ausblick von der Terasse}
\end{figure}

Die Arbeit ist relativ hart, aber abwechslungsreich.  Sogar auf dem
Markt verkaufen durfte ich! Sehr spannend. Aus vielerlei Perspektiven
zählen Will und Gerrit zu den besten WWOOfing Hosts, bei denen ich das
Glück hatte, aufgenommen zu werden. In Konjunktion mit meiner etwas
merkwürdigen und gestressten Stimmung in den letzten Wochen muss ich
aber auch gestehen, dass ich die Sache etwas ambivalent sehe.  Diese
Ambivalenz hat mir in letzter Zeit sehr viel zu denken gegeben.

Wir unterhalten uns wunderbar und sehr lang zu und nach den meisten
Mahlzeiten, dennoch sind die Hosts eher gut, aber nicht ``warm''. Das
mag von ihrer halb professionellen Einstellung gegenüber WWOOFern
liegen, wobei ich damit, nun da ich weiss, dass der Garten und das
Olivenöl wirklich nur Hobby sind, besser klar komme. Will und
besonders Gerrit sind schon über das Berufsleben hinaus (Gerrit ist 69
Jahre alt, ich habe ihn auf Mitte 50 geschätzt), brauchen die Arbeit
scheinbar aber doch, denn besonders Gerrit arbeitet bis zum
buchstäblichen Umfallen\footnote{Inklusive Krankenhausbesuch!}.  Von
uns wird das nicht erwartet, aber dennoch spiegelt sich das in einer
gewissen Erwartungshaltung wider. In der Praxis erfährt man meistens
nur, wenn etwas falsch ist und muss Lob ``erfragen''. Das alles hat
sich wahrscheinlich durch die schiere Masse an WOOFern, die hier über
das Jahr arbeiten, so eingependelt und ist nun einfach
hinzunehmen. Damit ist es auch schwieriger motiviert und effizient zu
arbeiten, da einem immer die Angst vor dem Fehler im Nacken sitzt. Um
einen Schluss damit zu machen: Es sind die ersten Hosts, bei denen ich
mich in der schwachen Position des Bittstellers sehe. Daneben aber ist
alles- und besonders das Essen - tiptop! Jeder WWOOFing Host ist
anders und das ist auch gut so!

Nun, zu entspannen - das ist so eine Sache. Ich habe mich wohl etwas
in eine ``Ich muss meine ToDo-Liste abarbeiten, es so viel zu tun'' -
Stimmung hineingesteigert. Und da mir hier, weil ich endlich mal etwas
unternehme und wir so lang am Esstisch reden, erstaunlich wenig Zeit
bleibt, kann das sehr frustrierend werden. Ich sage mir jetzt: du
kannst nur das tun, was du auch wirklich jetzt tun kannst. Nun, das
klappt mal mehr und manchmal weniger, aber die Tendenz sieht gut aus.

Ich WWOOFe hier nicht allein. In den ersten zwei Wochen gab es noch eine
Kiwi WWOOFerin in den 40igern namens Tracy. Und Tracy war und ist
wirklich das Beste hier. Unglaublich großherzig, humorvoll und auch
tiefsinnig wurde sie mir zur guten Freundin, so gut, dass es nur mit
Micha zu vergleichen, nicht aber in Worte zu fassen ist. Tracy selbst
ist zwar viel gereist, war nebenbei aber Work-A-Holic und Mutter. Um mal
auszusteigen ist Sie geWWOOFt und schließlich hier gelandet. Das eigene
Land zu bereisen ist eine gute Idee. Nun, jetzt weiß ich, was ich mache,
wenn ich zurück in Deutschland bin.
\begin{figure}[h]
  \centering
  \includegraphics[width=\textwidth]{13/tracy.JPG}
  \mycap{Tracy}
\end{figure}

Das wunderschöne Whakatane ist eine sehr offene, kleine aber schöne
Stadt und so verbrachte ich meine erste Woche hier damit zu arbeiten,
mir Sorgen zu machen und die Stadt zu bewundern.

Will und Gerrit schlugen eines Abends vor, wenn man schon einmal in
der Gegend sei, die Coromandel Halbinsel zu besichtigen (einen der
schönsten Landstriche Neuseelands). Ich, immer noch meschugge vom
Ankunfts-Schock (irgendwie hab ich den bei neuen Hosts immer), legte
die Idee erst einmal zu den Akten, bis Tracy vorschlug, man könne doch
zusammen reisen. Also setzten wir uns ans Planen (ich hasse planen,
habe aber noch zwei Nachmittage mit dem Planen meines
Südinselaufenthaltes verbracht) und brachen bald darauf zum
wunderbaren 4-Tages-Trip auf. Und wieder hatte ich großes Glück,
Tracys wunderbaren Bruder, dessen Frau und weitere Freunde kennen zu
lernen, bei denen wir übernachten durften. Es war eine gute Erfahrung,
zu sehen, wie viele warmherzige Menschen es auf der Welt gibt. Besagte
Freunde von Tracy waren in ihrer Kindheit wie zweite Eltern für sie
und somit waren viele Erinnerungen mit dem Besuch und dem wunderbaren
Stück Land, auf dem sie oft spielte (an der Formulierung ist noch zu
arbeiten), verbunden. An Sommertagen als 13- Jährige spontan auf dem
Meer drauf los zu segeln, das klingt für mich traumhaft und
unvorstellbar. Auch in anderen Hinsichten haben wir einen lohnenden
Trip verbracht. Das meine ich buchstäblich, denn die Aussicht war
wundervoll und ich durfte aus dem Fenster gaffen (und filmen, Verweis
auf Google-Photos), während Tracy halsbrecherisch im Kiwistyle
fuhr.
\begin{figure}[h]
  \centering
  \includegraphics[width=\textwidth]{13/marmite.JPG}
  \mycap{Hier habe ich zum erstenmal Marmite zu sch\"atzen gelernt.}
\end{figure}

Um es kurz zu machen: wir sind einmal rundherumgefahren und haben viel
gesehen. Dabei habe ich gelernt, dass Touristenattraktionen einfach
lächerlich sind und man schon mit ein paar wenigen Schritten in die
Natur für sich selbst und mit guten Menschen noch viel Schöneres
erleben kann. Nun ein Gutes hat es dann doch gehabt: den Touris am
Hot-Water-Beach dabei zuzuschauen, wie sie sich, Schulter an Schulter
stehend, gegenseitig die Sandlöcher zuschaufeln, war schon mit
erheblichen Amusement verbunden. Auch war der Anblick von Mount
Maunganui atemberaubend.
\begin{figure}[h]
  \centering
  \includegraphics[width=\textwidth]{13/maunganui.JPG}
  \mycap{Ausblick von Mount Maunganui}
\end{figure}

Nun bin ich wieder zurück und muss wieder einmal gestehen, das ich trotz
der wunderbaren Reise froh bin, wieder Back-To-Normal zu sein (was auch
immer das beim WWOOOFing bedeuten mag).

Tracy ist weitergezogen, hilft ihrer Schwester beim Einrichten eines
Kindergartens und wird, hinter ihrer Tochter her, nach Asien (Cambodia,
Laos, etc\ldots) reisen. Ich indessen vermisse sie sehr, komme aber in
den Genuss, jetzt einmal den Erfahrenen spielen zu dürfen.

Das bedeutet, dass wir eine neue dänische WWOOFerin haben, mit der ich
mich schon recht gut angefreundet habe. Sie ist Psychologie- und
Neuro-Sciences-Studentin und nimmt sich eine Auszeit vor ihrem
Master-Studium. Nun heisst es ihr die Neuseeländischen Verfahrensweisen
näher zu bringen und Erfahrungen weiter zu geben. Ist auf jeden Fall
sehr spannend für beide Seiten.

Es ist erstaunlich, wie gut ich schon zurechtkomme (immer noch
entfernt vom Optimum). Leute kennenlernen, im Supermarkt oder dem Hot
Pool mit Wildfremden Freundschaften zu schließen oder auf Mount
Manganui mit einem Tschechen ohne großes Brimborium ins Gespräch zu
kommen, all das wäre für mich vor einem halben Jahr wohl noch nicht
möglich gewesen. Nun, ich hatte wohl keine Ahnung, worauf ich mich
einließ und das bekomme ich auch zu spüren, aber es lohnt sich. Ich
bin nicht frei von Zweifeln, was das WWOOFen betrifft, aber ich komme
immer besser zurecht und es steht mir immer noch offen, etwas anderes
zu machen, auch wenn mir der aktuelle Modus Vivendi sehr gefällt.
Merkwürdiger Weise lobt jeder mein Engisch\ldots{} nun ja, das Lernen
einer Fremdsprache ist hier nicht so selbstverständlich, wie in
Deutschland.

Damit gab es mal eine grobe Zusammenfassung und ich falle ins Bett.
Heute war Markttag und ich bin geschafft. :P
