\chapdate{19.08.2016}
\chapter{Statusbericht}

Nun bin ich schon vier Wochen in Neuseeland.

Ich habe einen interessanten und anspruchsvollen Job beim Department of
Conservation (freundlicherweise vermittelt durch Matt). Das DoC streamt
die Entwicklung eines Albatros-Jungen live auf YouTube und ich darf die
Methode und Hardware dokumentieren und optimieren, bzw. Software dafür
entwickeln. Meine Arbeit trägt Früchte: der Stream muss schon seit zwei
Tagen nicht täglich viermal (oder noch öfter) manuell via TeamViever neu
gestartet werden. Auch habe ich ein simples Übergangswebinterface (mit
einem Relay Server!) geschrieben. Good bye Firewalls. Ich habe gestern
'zu lang' gearbeitet. So etwas kann in Neuseeland vorkommen!


Wenn die Sache vorüber ist fange ich mit dem WWOOFing\footnote{World
Wide Opportunities on Organic Farms. Arbeit f\"ur Kost und Logie auf
Farmen oder in G\"arten.} (jetzt wohl doch auf der Nordinsel)
an. Vielleicht schließt sich arbeitstechnisch auch noch was an\ldots{}
Je nachdem wie lange mich Edith und Matt noch aushalten bleibe ich
vorerst in Wellington.

Das mit den Bildern versuche ich ich noch hinzubekommen. Die neusten
sind aus Zealandia. Ich weiß nun, warum Neuseeland so reich an
endemischen Spezies ist :).

Jaja ich muss mal einen RSS feed für die Sache einrichten :)

\begin{figure}[h]
  \centering
  \includegraphics[width=\textwidth]{04/zealandia_bird.jpg}
  \mycap{Ein Kaka in Zealandia.}
\end{figure}
