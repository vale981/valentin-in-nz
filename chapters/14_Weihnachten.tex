\chapdate{25.12.2016}
\chapter{Weihnachten}

Frohe Weihnachten euch allen. Zu guter Letzt hat sich vorgestern auch
bei mir eine weihnachtliche Stimmung eingestellt. (Auch dank Mamas
Lebkuchenpacket. Danke :P.)

\textgreater{} Und so begab es sich, dass Valentin, Sohn des Stefan
(Sohn des Otto), über eine Straße, die das Volk der Neuseeländer zu
jener Zeit State-Highway-One nannten, nach Wellington, der Wohnstätte
der Familie der Robertsons, zog. Aber es kamen ihm allerlei Sorgen und
Zweifel dabei. Jedoch als er sah, dass die Lande, an denen er vorbeizog,
der Heimat {[}zunehmend{]} ähnlich sahen, so wusste er, dass er dem
Hause des Matt und der Edith nahe war. Es ward wie ein Licht in seiner
Seele und er rief aus Halleluja und er pries den Herrn zum Feste der
Geburt Jesu mit Freunden und nicht in Einsamkeit zu sein.

Nach meinem letzten Blogeintrag führte die ganze Situation zu einer
kleinen, mehr oder weniger produktiven Aussprache mit meinen WWOOFing
Hosts. Um es zusammenzufassen kann man wohl sagen, dass wir uns etwas
falsch verstanden haben und ich insbesondere die Kritik des brummigen
Hosts zu streng nahm. In der Folge habe ich versucht, mich nach bestem
Willen zu verbessern, war jedoch weiterhin das Greuelventil für den
überarbeiteten Gerrit. Wilhelmina war jedoch so freundlich, mir dann
doch immer einmal zu signalisieren, dass ich nicht ganz so schlimm für
die beiden bin, wie ich vielleicht annahm. Auch die für die
Weihnachtszeit angereiste Tochter Kina trug zur Entspannung der Hosts,
und damit auch zur Entspannung meiner Situation, bei. Schlussendlich bin
ich dann am 19. Dezember im Guten und mit guten Erinnerungen
aufgebrochen, reich beschenkt mir einer Flasche Olivenöl und einem Glas
Honig.

Aufgebrochen zu einer wunderbar interessanten Reisewoche. Ich, von mir
aus, hätte wohl die letzte Woche vor Weihnachten einfach noch einmal
geWWOOFt und habe es Ediths Aufmunterungen zu verdanken, mich zu einer
kleinen Rundreise über die Ostküste bis zum Tongariro National Park
aufgerafft zu haben. Es brauchte einen arbeitsamen, aber sehr
interessanten Nachmittag und die Route war ausgeplant und die Hostels
waren gebucht.

Nach einer langen, aber sehr pittoresken Fahrt um das East Cape, auf dem
der östlichste Leuchtturm der Welt steht und bei dem ich zwei nette
deutsche Radler traf, wurde ich äußerst positiv von meinem Hostel
überrascht. Nicht allein waren die Umgebung und die Einrichtung
wunderschön, nein auch bekam ich kostenfrei, aufgrund von Unterfüllung,
ein Einzelzimmer mit Sonnenaufgangsblick, den ich, da ich ganz ohne
Wecker um 5 Uhr am Morgen erwachte, alsbald genießen durfte. (Um ehrlich
zu sein: die Sonne versteckte sich hinter einer Wolke, war also gar
nicht direkt zu erkennen, aber das Farbenspiel war dennoch sehr
ansehnlich.) Am nächsten Morgen war ich bereits auf dem besten Weg, nach
Gisborne weiter zu fahren, kam aber zu meinem Glück, dass mich wohl die
ganze Woche verfolgte, mit einem Schweizer Radreisenden ins Gespräch.
Ich entschied, noch eine Nacht im Hostel zu verweilen und brach zusammen
mit dem Schweizer zu einem sehr lohnenswerten Tagesausflug auf. Der
East-Coast scheint sehr beliebt unter Radfahrern zu sein, sodass es im
Hostel neben Anraud auch noch zwei niederländische und einen britischen
Radfahrer gab. Zurück zum Faden: Arnaud und ich wanderten also zu Cooks
Cove, einer kleinen Bucht, die Captain Cook bei seiner Umsegelung
Neuseelands entdeckt, und als besonders und außergewöhnlich schön
befunden hat. Und auch wir konnten diesem Urteil nur zustimmen, bot die
Bucht doch einen Anblick, wie ein Photo aus dem Reisemagazin. Sogar im
eiskalten Wasser konnten wir planschen. Danach haben wir uns noch den
längsten Anleger in der östlichen Hemisphäre (jaja der Begriff ist
inadäquat\ldots) angesehen und durchlaufen. Der besagte Anleger stammt
noch aus der Zeit nach dem Weltkrieg, als man in Neuseeland die Schafe
und Rinder zum Hafen trieb und direkt geschlachtet auf Kühlbote lud, um
das verwüstete Europa zu versorgen. Besonders ausgeprägt war diese
Verfahrensweise am East-Coast, der als ganzer Landzug bis weit ins
Inland eine einzige Farm ist. Es gibt in Neuseeland siebzig Millionen
Kühe, Rinder und natürlich Schafe auf viereinhalb Millionen Menschen und
trotzdem sind Milch und Fleisch teuer. Das liegt, wie mir vom
sympathischen Hotelbesitzer erklärt wurde, am wunderbaren, komplett
freien Handelsmarkt in Neuseeland. So verkauft man lieber im Export und
wer im eigenen Lande auch noch etwas abhaben möchte, der zahlt doch
bitte dieselben hohen Preise. Es gibt hier keine Zuschüsse und keine
Unterstützung, sodass den Farmern nichts anderes übrig bleibt, als
mitzuspielen, um im Geschäft zu bleiben.

Da mir das nicht genug Aktivität für den Tag war und es mir nach
Abenteuer (Querfeldeinmarsch) stand, habe ich am Abend noch den Hügel
hinter dem Hostel erklommen. Mein Ehrgeiz peitschte mich bis zehn Meter
unter den Gipfel, den ich dann aber im Angesicht eines Geröllhanges zu
meiner Linken und Felsblöcken zu meiner Rechten nicht mit Sandalen an
den Füßen beklettern wollte. Auf dem Weg nach unten beschloss ich einen
scheinbar direkteren Weg zu nehmen, endete im Dickicht und musste
umdrehen, um nach einer anderen Route zu suchen. So habe ich gelernt:
Nimm immer den Weg zurück, den du gekommen bist. (Denn du weißt, dass er
funktioniert.) Aus einem zwanzigminütigen Spaziergang wurde also eine
zwei Stunden Wanderung. Auch die Blasen, die ich mir in meinen
Wanderschuhen beim Austragen von Werbezettelchen für meine Hosts (30km
in zwei Tagen) gelaufen habe, dankten es mir. Zum Abend kochte ich mit
Arnaud ein paar Nudeln, die wir dann zusammen mit zwei frisch
angekommenen und recht planlosen deutschen Mädels (auf die meisten
unserer Fragen gaben sie dieselbe Antwort: ``Wir wissen {[}es{]}
nicht\ldots'') verspeisten.

Am nächsten Tage ging es schließlich weiter zum Tongariro National Park.
Einen Zwischenstopp machte ich in Gisborne, um mir im dortigen Park ein
wenig die Füße zu vertreten, eine Statue von Captain Cook zu bewundern
und das Östlichste Observatorium der Welt anzusehen (Naja, eben nur ein
kleines weißes Haus mit Kuppel :P.). Im Anschluss daran durfte ich auf
einer Sechsstündigen Fahrt allerhand schöne Natur bewundern und legte
mich im Hostel nach einem kleinen Abendbrot direkt schlafen.

Um fünf Uhr in der Frühe peitschte ich mich am folgenden Tage aus dem
Bett, um das Shuttle zur Tongariro Alpine Crossing zu erwischen. Ja,
auch ich habe mich mal wieder wie ein Tourist benommen und bin die
berühmte 19 Kilometer lange Crossing gewandert. Trotz der den Blick
versperrenden Wolken habe ich Ansichten genossen, die mich erstaunten
und die wohl in ihrer Unwirklichkeit unvergleichlich mit allem bisher
Gesehenen waren. Und trotzdem verspürte ich eine Ambivalenz, fühlte ich
mich doch aufgrund der schieren Massen anderer Wanderer, die auf dem
Wege vor und hinter mir mehr oder weniger motiviert marschierten, sehr
gewöhnlich. Nachdem ich den großen Anstieg, der uns gleich am Anfang
erwartete, fast rannte und viele überholt habe, traute ich mir zu, den
in Wolken verhüllten Ngauruhoe (Mt. Doom aus TLOTR) zu besteigen. So
machte ich mich zusammen mit einem freundlichen Briten an den Aufstieg.
Als sich die Sicht dann aber auf einige Meter beschränkte und ich in der
Aussicht, einen Geröllhang zu erklettern, zunehmend die Nerven verlor,
beschloss ich umzukehren und meine Kräfte für die verbleibenden 12
Kilometer auf der Crossing aufzusparen. Derselben Ansicht waren zwei
junge Damen, denen ich mich für eine Weile der Wanderung anschloss.
Gesellschaft ist manchmal eben doch dem einsamen Vor-sich-hin-grübeln
vorzuziehen. Der weitere Verlauf meiner Wanderung lässt sich besser
photographisch beschreiben und ich verweise hiermit wieder einmal auf
meine Photofreigabe. Nachdem mich über die letzten Kilometer die Massen,
die ich zuvor überholte, ihrerseits überholten, weil meine mit Blasen
übersäten Füße so furchtbar schmerzten, ging es zurück ins Hostel. Um an
Toast zu sparen, kochte ich mir Pfannkuchen mit einer herzhaften und
sehr schmackhaften Füllung und auch am Folgetag fand ich große Freude an
der Kocherei und versuchte meine Vorräte möglichst effizient zu
verkochen (Spiegelei mit der restlichen Füllung und Crêpes als
Toastersatz :P).

Am 23. Dezember erwischte ich die letzten sonnigen und regenfreien
Stunden, um zu den überaus ansehnlichen Tarnaki Falls zu wandern, wobei
man sowohl die typische Tongariro Steppe (mit Blick auf die Vulkane),
als auch den grünen Native-Bush bewundern durfte. Witzigerweise waren
wir fast genau vor drei Jahren schon einmal in der Gegend und mir stand
im erstklassigen Museum und Visitor Centre Vorort ein Deja vu bevor.
Sowohl die Wanderung als auch das Visitor Centre befinden sich nahe des
Whakapapa Village, eines von Ski-Enthusiasten gegründeten Feriendorfs
mit allerlei Restaurants, Cafes und Unterkünften (unter ihnen auch das
berühmte Baudenkmal und Skihotel Chateau Tongariro, endlich einmal ein
richtiges Steingebäude!). Aus Neugier fuhr ich zu guter Letzt auch die
Straße zum Skigebiet hinauf, um inmitten von Nebel, Regen und Wolken
Skilifte und Felsklippen zu bewundern (sehr surreal).

Das Weihnachtsfest mit Edith und Familie war sehr harmonisch und
gemütlich, sodass ich es endlich einmal geschafft habe, richtig zu
entspannen. Ich kann mich glücklich schätzen, so reich beschenkt worden
zu sein (Danke Mutti und Papi und Omi und alle anderen ;)!). Auch das
Weihnachtsabendessen im `München', einem deutschen Restaurant, schmeckte
überaus gut. (Ich habe irgendwie das Talent, immer den größten Appetit
mitzubringen und die kleinste Portion abzubekommen. :P)

Am Weihnachtstage dann ging ich (zum ersten Mal seit langem) in die
katholische Kirche in Khandallah und musste feststellen, dass selbst ich
die Gemeindegemeinschaft doch sehr vermisst habe. Auch die Predigt des
humorvollen Pfarrers zum Thema ``Ist Religion eine Ausflucht'' (Sie ist
keine, sie ist eine Hilfe \ldots{} ein Mittel gegen spirituelle Armut
\ldots) war zugegebenermaßen sehr interessant.

Punkt. :) Die nächsten Tage werden hoffentlich sehr entspannt :).

Eine Frohe Weihnacht und vielen Dank für eure Geduld.
