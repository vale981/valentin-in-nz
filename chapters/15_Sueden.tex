\chapdate{12.01.2017}
\chapter{S\"uden}

Grüße von der Südinsel.

Mit einem tollen Blick auf das Gebirge im Norden der Südinsel verfasse
ich mal wieder einen kleinen Bericht für euch. Da ich zur Zeit mal
wieder dabei bin, neue Berge zu erklimmen, werde ich mich etwas kürzer
fassen, als es in Anbetracht der seit dem letzten Post vergangenen Zeit
vielleicht zu erwarten wäre.

\begin{figure}[h]
  \centering
  \includegraphics[width=\textwidth]{15/onferry.JPG}
  \mycap{Auf der F\"ahre}
\end{figure}
Nachdem mein Auto, in dem ich klugerweise ein Licht über die Nacht
brennen ließ, nach einer in aller Frühe durchgeführten Starthilfe mit
dem fünften Versuch dann doch noch startete, habe ich die Fähre zur
Südinsel noch erwischt und bin nach einer langen Tagesreise bei meinem
neuen WWOOFing Host angekommen. Irgendwo im Nirgendwo auf einem kleinen
Hügel liegt ein kleinen Bed and Breakfast, in dem ich nun einen
phänomenalen Ausblick genießen kann. Reg Turner, mein Host, hat die Idee
der Luxus-Lodges überhaupt erst nach Neuseeland gebracht und sich jetzt
hier zur Ruhe gesetzt. Ich kam nach meiner letzten WWOOFing Erfahrung
mit etwas verschobenen Erwartungen auf die Südinsel, nur um zu erkennen,
das Whakatane wohl eine Ausnahme darstellt. Mit Reg ist es ein ganz
anderes Gefühl. Die eigene Arbeit wird gewürdigt, Initiative begrüßt und
vor allem werden Fehler verziehen.
\begin{figure}[h]
  \centering
  \includegraphics[width=\textwidth]{15/lodgeview.JPG}
  \mycap{Ausblick der Lodge}
\end{figure}

Ich wohne in einem kleinen Bungalow neben der Lodge und habe die ersten
Nachmittage damit verbracht, denselben ein wenig zu säubern und
herzurichten. Nichts Gravierendes, aber man möchte es ja gern ein wenig
wohnlich haben. Ich genoss also die ersten Tage, allein zu sein und
meinen Bungalow für mich zu haben. Doch bevor ich mich von der
Gesellschaft abnabeln konnte, schneite ein französischer WWOOFer herein.
Welch ein Glück, denn zu mehreren macht WWOOFen immer mehr Spaß.

\begin{figure}[h]
  \centering
  \includegraphics[width=\textwidth]{15/bread.JPG}
  \mycap{Reg und Meddy beim Brotbacken}
\end{figure}
Meddy ist Bäcker und buk zu unserer großen Freude gleich am ersten Tag
ein wunderbares Brot. Fasziniert von dieser Kunst bat ich darum, mich
auch einmal an einem Brot versuchen zu dürfen. Gesagt, getan: gestern
habe ich schon mein drittes Brot gebacken und seitdem ich herausfand,
wie schön das europäische Brot doch ist, bisher kein Toastbrot mehr
angerührt. Das Brotbacken nimmt erstaunlich viel Zeit in Anspruch, ist
aber aufgrund der kreativen Freiheiten (Gemüse in's Brot backen :P)
eine sehr interessante Beschäftigung.

Zwei Tage nach Meddys Ankunft waren wir dann schon vier WWOOFer. Zwei
Deutsche sind zu uns gestoßen und wir sind nun eine eifrige Task-Force
für den Sommer-Cleanup. Ich selbst habe die letzten Tage, nachdem
zuerst aufgrund des Regenwetters Hausarbeit angesagt war, die etwas
abenteuerlich steile Auffahrt mit dem Weedeater\footnote{Motorsense}
gemäht. Heute dann haben wir die Garage einmal gründlich aufgeräumt
und durchetikettiert.

Erstaunlicherweise habe ich schon am zweiten Tag frei bekommen und
daraufhin versucht Mount Stevens zu besteigen. Auf halbem Wege zum
Gipfel fiel mir dann aber auf, dass ich zwar mein Wasser sehr
vorausschauend aufgefüllt, aber nicht eingepackt hatte. Also kehrte ich
um und das zu meinem Glück, denn der Berggipfel war auf einmal in
bedrohlich dunkle Wolken gehüllt. Auf dem Rückweg motivierte ich dann
noch eine ganze Reihe von Unentschlossenen in das eiskalte Flusswasser
zu springen, indem ich mit gutem Beispiel voran ging.
\begin{figure}[h]
  \centering
  \includegraphics[width=\textwidth]{15/bath.JPG}
  \mycap{Die Badestelle}
\end{figure}

Ein paar Tage später wanderte ich zu ein paar Höhlen (Grüße an Firouz
und Familie!)  und traf einen amerikanischen Reisenden aus Australien,
mit dem ich mich prächtig über dies und jenes unterhielt und den ich
schon bald als Freund und Seelenbruder gewann. Da ich unmöglich alle
Kontaktdaten meiner überaus glücklichen Begegnungen in Neuseeland
festhalten kann, habe ich jetzt eine neue Datei eröffnen
müssen. Erstaunlich, wie viele tolle Menschen es gibt.
\begin{figure}[h]
  \centering
  \includegraphics[width=\textwidth]{15/kyle.JPG}
  \mycap{Kyle vor der H\"ohle}
\end{figure}

Ich war sehr glücklich, als Kyle, so der Name der Wanderbekanntschaft,
mir erzählte, dass ihm bei Weka-Arts ähnliches widerfahren ist, wie
mir. Weka-Arts, die ich auf einer kleinen Fahrradtour entdeckte,
ist eine tolle Galerie von sehr ansehnlicher Holzarbeit. Wenn man das
Grundstück, auf dem sich diese Ausstellung befinden soll, betritt,
strahlt einem ein großes, rotes Schild: `OPEN' entgegen. Davon
eingeladen fange ich also an, durch den Garten, auf den die Einfahrt
führt, zu schlendern und eifrig photographisch zu dokumentieren, wie
schön der Ort doch sei.
\begin{figure}[h]
  \centering
  \includegraphics[width=\textwidth]{15/weka.JPG}
  \mycap{Weka Arts, der Ort meines ``Verbrechens''}
\end{figure}

Nach einer Weile kommt dann der Besitzer zu mir
herüber und fragt mich, wer ich denn sei und warum ich denn einfach so
in fremder Leute Gärten herumschlendere. Ganz perplex antworte ich
ehrlich und wenig gewitzt, dass ich wohl von dem Schild in der Einfahrt
verwirrt gewesen sein musste und ich normalerweise nicht die Gewohnheit
pflege, Grundstücke als öffentlich zu betrachten. Ich hätte natürlich
behaupten können, dass der sehr schön angelegte Garten an sich doch
schon ein Kunstwerk oder eine Galerie, wie sie auf dem Eingangsschild
beworben wurde, darstellt.

Wie auch immer. Der Herr erwiderte dann, dass er nicht hinter
Schildern und Zäunen leben möchte und deshalb seinen Garten nicht als
privat markiert hatte. Warum er aber vor seinem Haus eine Kette mit
dem Schild `Private' aufgehängt hatte, war mir dann nicht ganz
klar. Vielleicht sind Ketten O.K. oder er hat Spaß daran, harmlose
Touristen in die Irre zu führen, die annehmen, dass alles als nicht
``Privat'' Gekennzeichnete erlaubt ist. Mit dem Schrecken habe ich mir
dann noch die eigentliche Ausstellung angesehen. Interessante Möbel
von Brettchen, verziert mit kleinen Holzpilzen, über Lampenständer bis
hin zu verrückten Tischen und Schränken waren zu bestaunen. Falls also
jemand Interesse an einem schön verzierten Holzlöffel hat, so melde er
sich jetzt!

\begin{figure}[h]
  \centering
  \includegraphics[width=\textwidth]{15/spit.JPG}
  \mycap{Sandsturm auf dem Farewell-Spit}
\end{figure}
Auch erwähnenswert ist der Ausflug zum Farewell Spit, den ich und Meddy
unternommen haben. Das Farewell-Spit ist die lange, dünne, aus einem
einzigen langen Strand bestehende nördlichste Landzunge der Südinsel,
die man sogar mit exorbitant teueren Tourbussen befahren kann. Wir sind
über die nächstgelegen Hügel gewandert und hatten einen erstaunlichen
Ausblick auf das von einem Sandsturm überrollte Spit. Ich habe an diesem
erstaunlich schöne Natur und noch viel schönere Strände sehen dürfen,
muss aber zugeben, das ich selbst in Wellington noch nie so einen Wind
erlebt habe.

Der Wind machte alles aber noch viel interessanter, denn
jeder hat Bilder vom Farewell Spit, aber wer hat schon Bilder von einem
Sandsturm auf der Landzunge? Die wandernden Dünen und blauen Wellen auf
dem Whariki-Beach zu betrachten, war auch eine sehr eindrucksvolle
Erfahrung. Gleich zwei Landschaftswahrzeichen an einem Tag! Abends dann
bin ich nach dem Brotbacken dann in mein Bett gefallen und erst gegen
zwölf eingeschlafen.

So weit so gut. Das waren die bisher südlichsten Abenteuer des Valentin
in einer (sehr,) sehr kurzen Fassung. Danke fürs einschalten und bis zum
nächsten Mal liebe Kinder :).

\begin{figure}[h]
  \centering
  \includegraphics[width=\textwidth]{15/dunes.JPG}
  \mycap{Ein vergleichsweise weniger windiger Strand}
\end{figure}
