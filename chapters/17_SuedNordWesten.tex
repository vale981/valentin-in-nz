\chapdate{10.02.2017}
\chapter{S\"ud Nord Westen}

Zumindest einen kleinen Bericht bin ich euch schuldig.

Meine verbleibenden Tage in Nelson waren wunderbar und wieder empfinde
ich große Dankbarkeit, war Cathy Jones doch wieder so gut zu mir. Wenn
man in der Gegend ist, dann gilt es unter Reisenden schon fast als
Sakrileg, den Abel Tasman National Park nicht zu bewandern.

\begin{figure}[h]
  \centering
  \includegraphics[width=\textwidth]{17/split_apple.JPG}
  \mycap{Split-Apple Rock}
\end{figure}
Da ich ein Greenhorn bin, habe ich mich einmal mehr für die Tagestour
entschieden.  Also stehe ich pünktlich um 6:30 Uhr auf, um dann 9 Uhr
gerade noch mit guter Not das Wassertaxi zu erreichen. Allein die
Schiffsfahrt lohnte schon des Ausflugs. Mit einem Affenzahn ging es
zuerst auf eine kleine Exkursion zum Split-Apple Rock, einem in der
Mitte gespaltenen, aus dem Wasser ragenden, kugelförmigen und sehr
apfelähnlichen Felsbrocken, und danach durch diverse Buchten, bis ich
in der Torrent Bay aussteigen durfte.  Unter anderem gab es auch
neuseeländische Pelzrobben zu bestaunen.

\begin{figure}[h]
  \centering
  \includegraphics[width=\textwidth]{17/bnw.JPG}
  \mycap{Der Tag beginnt in Graustuffen.}
\end{figure}
Meine Sorge, der Wanderweg würde von den Horden in den
Booten (die Wassertaxis waren bis auf den letzten Platz besetzt)
überrannt werden, wurde erst von mir genommen, als ich erfuhr, dass
alle Fahrgäste außer mir selbst bis ganz zum Anfang des Wanderweges
fahren (ich mache ja nur eine Tagestour). Einige Minuten später ging
mir dann auf, dass die Bote schon seit Tagen Hochkonjunktur feierten
und ich mich beim Wandern einer reichlichen Gesellschaft erfreuen
durfte. Und doch war es wie im Paradies (und das Optische ist ja
ausreichend photographisch dokumentiert und bedarf keiner weiteren
Erläuterung).

\begin{figure}[h]
  \centering
  \includegraphics[width=\textwidth]{17/path.JPG}
  \mycap{Palmenges\"aumter Wanderweg}
\end{figure}
Alle Traumstrände waren wie leergefegt. Kein Mensch, keine Robbe,
keine Sandfly. Alle Welt wandelte auf den Wegen, denn zum Baden gab es
zu viel \ldots{} \ldots.  naaa \ldots. Niederschlag! (Wer ist jetzt in
poetischer Stimmung?) Immer munter zog ich also ohne Angst vor
Sonnenbrand unter dem schützenden Wolkendach daher und ließ den Regen
hinter mir. (Als ich einmal den Fehler machte, hinter mich zu schauen,
jagte mir eine graue Regenwand einen Mordsschrecken ein!).

So wanderte ich also für meine ersten sechs Kilometer fröhlich vor
mich hin, bestaunte und entspannte. Plötzlich deutet eine Dame von
durchaus seriöser Erscheinung auf den nächstgelegenen Felsbrocken und
erklärt mir, dass ich da einen Dinosaurier sehen könne. Bevor ich
antworten kann, fährt sie fort, dass man weiter unten am Hügel noch
einen Wal erkennen könne und generell die ganze Küste aus allerlei
Versteinertem bestehe. Ich, der ich immer noch glaube, es gehe nur um
visuelle Ähnlichkeiten, möchte gerade einräumen, dass der zuerst
erwähnte Felsbrocken für mich wie ein Fisch aussehe, als mir die Dame
mit Überzeugung entgegnet, dass sie auf der Bootsfahrt (nicht auf
meinem Boot\ldots) Knochenstaub auf den Füßen hatte und nur
Dinosaurier und Wale, nicht aber Fische dieselben aufweisen.
Danach wünscht sie mir einen schönen Tag und zieht schnurstracks von
dannen.

Ich bin mir immer noch nicht ganz sicher, ob irgendeine Art Spaß mit
mir getrieben wurde, hätte aber gern entgegnet, dass sich für allerlei
unverstandene Dinge allerlei mehr oder weniger plausible Erklärungen
finden lassen können. Vielleicht sollte mir das zeigen, dass jeder,
der nur genügend Selbstbewusstsein besitzt, den größten Humbug von
sich geben, dabei aber immer überzeugend und seriös erscheinen kann.

\begin{figure}[h]
  \centering
  \includegraphics[width=\textwidth]{17/gumpe.JPG}
  \mycap{Gumpe mit badenden Wanderern}
\end{figure}
Nach meiner Mittagspause fühlte ich mich miserabel und begann daran zu
zweifeln, dass ich, wenn ich mich nach schon 6 Kilometern so schlapp
fühle, die restlichen 14 noch schaffen kann. Zwei Kilometer später wies
auf einmal ein kleiner Wegweiser auf eine kurze Abzweigung (500m) zu
Cleopatras Pool hin. Keine zwei Kilometer, wie ich irrtümlicherweise in
meine Gratis-Karte hineininterpretiert hatte. Eine echte Gumpe, in
Korsika Qualität: Phänomenal und dann zeigt sich auch, zum einzigen mal
an diesem Tag, der Sonnenschein. Nichts wie \ldots{} \ldots{} in's
Wasser (ätsch, schon wieder nicht gereimt). Wirklich kalt, aber ebenso
erfrischend!

\begin{figure}[h]
  \centering
  \includegraphics[width=\textwidth]{17/fat_tui.JPG}
  \mycap{The Fat Tui}
\end{figure}
Nach dieser kleinen Planscherei verging der Rest der Wanderung durch
die fast schon monotone Schönheit des Abel Tasman Parks wie im
Fluge. Zum Abendbrot gab es nach einer durch enorme Nachfrage
bedingten halbstündigen Wartezeit einen überaus bemerkenswert
schmackhaften Burger aus dem Fat-Tui Food-Truck.

An meinem letzten Tag in Nelson war ich noch einmal in der Suter Art
Gallery und habe wieder nur einen Raum geschafft, weil man schon um 4:30
Uhr schließt! Auf der Suche nach einer neuen Mechanik für meinen Bass
bin ich dreifach am Musikladen vorbeigefahren. Danach schien mir das
Glück hold zu sein, so gab es tatsächlich einzelne Mechaniken zu kaufen.
Aber immer waren die Tuner für die falsche Seite, aus welchen Ecken der
Verkäufer Sie auch hervorzauberte (und der dieser Ecken gab es viele).
Danach bin ich aus Zufall noch einem Schild zum ``Center of New Zealand''
gefolgt und hatte einen tollen Ausblick auf Nelson und das quietschblaue
Meer.
\begin{figure}[h]
  \centering
  \includegraphics[width=\textwidth]{17/center.JPG}
  \mycap{Ausblick auf Nelson}
\end{figure}

Jetzt bin ich am Westcoast und schreibe diesen Blogpost im gemütlichen
Sofa des netten Hosts. Ich wohne hier einmal mehr irgendwo im
Nirgendwo und wir haben nur Solarstrom und Regenwasser. ``Nur'' ist
vielleicht zu kurz getreten, denn wir kommen damit ohne große
Limitierungen über die Runden und ich bin erstaunt, wie wenig
Solarpanele er auf dem Dach hat.

Schon an meinem ersten Tag wurde mir eröffnet, dass man (John, der
Host, sein Freund Michael und die 3 anderen WWOOFer) am Wochenende
einen Campingausflug in die Berge antreten wollte, um den Weg mit
Sägen und Scheren wieder gangbar zu machen und zu markieren. Hurra
\ldots{} soll ich jetzt in Freude oder Angst ausbrechen? Ich habe noch
nie in der Natur gecampt \ldots{} will ich diese Erfahrung überhaupt
machen? Ich nahm die Herausforderung an und so ging es 5:30 in der
Frühe los und ab in den Bush! Motivierende Sprüche wie: ``Das Gefühl,
Durst zu haben, ist nichts schlimmes'' (im Angesicht unserer begrenzten
Wasservorräte) brachten uns schon einmal in die rechte Stimmung :).

Zusammenfassend ausgedrückt muss ich eingestehen, dass der Trip
schrecklich grausam, aber lehrreich und eine tolle, besser nicht zu
wiederhohlende Erfahrung war. Selbst der ``professionelle'' und
abgehärtete Host John, der als Arzt schon in Afghanistan und am Südpol
war, musste zugeben, dass der Trip wohl eher ``extrem'' war.
\begin{figure}[h]
  \centering
  \includegraphics[width=\textwidth]{17/foulwind.JPG}
  \mycap{Eine Ahnung der S\"udalpen in der Foulwind Bay.}
\end{figure}

Im Grunde sind wir zwei Tage lang klitschnass einen Berg hinauf
(leider nicht ganz bis zum Gipfel) und danach eben wieder hinab
gestiegen. Dabei hatten John und Michael den Zustand des Tracks an
beiden Tagen etwas sehr optimistisch eingeschätzt. Da meine Regenjacke
leider nicht wasserdicht war und ich zu wenig Wechselsachen eingepackt
hatte, war ich wohl eher selbst schuld an meinem Unglück. Der sonnige
Abend auf einem Hügel auf halbem Weg bergauf (unserer ``Camp-Site'')
belohnte die Mühe mit tollen Ausblicken, \textbf{\textbf{Trockenheit}}
und einem gewissen Siegesgefühl.

Während der letzten Tage habe ich den Westcoast auf weniger dramatische
Weise erforscht und sehr viel Schönes gesehen. Die Fotos werden folgen,
sobald ich wieder eine gute Internetverbindung habe.

Bis dahin: Alles Gute und danke für's Lesen.
