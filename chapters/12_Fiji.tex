\chapdate{23.11.2016}
\chapter{Fiji}

Frisch aus dem Urlaub im Urlaub. Ich grüße von Fiji, denn ich schreibe
diesen Eintrag schon auf der Insel und veröffentliche ihn erst jetzt.
Ein paar wunderbare und sehr komfortable Tage waren es. Wir wohnen hier
in einem sehr schönen Ferienhaus mit Pool, Meeresblick (bzw.
Sonnenuntergangsblick) und erfrischender Brise zur Abendstunde. Viel
Entspannung und viel Freizeit. Das ganze erinnert mich an Gozo mit ein
bisschen mehr grün, aber der gleichen Hitze. Wir haben auch zwei
``Bedienstete'', die das Haus in Ordnung halten und kochen. Auch wenn
sie für Bezahlung arbeiten, so kann ich es doch nicht ab, bedient zu
werden als stände ich über anderen. Nun überfällt mich also immer ein
gewisses Unbehagen, wenn ich sie arbeiten sehe und ich versuche ab und
an zu helfen. Als wir am ersten Tag in die Stadt fuhren, um einzukaufen,
durfte ich erfahren, was ein echter Markt ist: viele kleine Stände mit
frischem Gemüse und allerhand interessanten Kleinigkeiten. Um die
nötigen Preisverhandlungen kümmerte sich unserer lokaler Führer Stanley.
Auf dem Weg zurück fiel mir dann auf, wie arm das Land Fiji ist. Der
Großteil der Bevölkerung lebt in Wellblechhütten und unsere
``Bediensteten'' schätzen sich mit einen überdurchschnittlich hohen
Monatslohn von umgerechnet weniger als 300 Euro glücklich, wobei die
Lebensmittelpreise auch gesalzen sind. Da ich gerade die Beweismethode
der vollständigen Induktion verstanden hatte, suchte mein Geist nach
einem neuen Problem und so stürzte mich die Ungleichheit auf der Welt in
eine tiefe Verzweiflung. Wie kann es sein, dass ich so ein Glück habe
und in Fiji auf einem Hügel (ja, auch im geographischen Sinne) über den
in Armut lebenden Urlaub mache. Wie kann es sein, dass ich mir dieser
Ungerechtigkeit bewusst war und dass sie mich aber nicht schon früher
zur Verzweiflung getrieben hat. Wenn nur die geringste Möglichkeit
besteht etwas ändern zu können, warum sollte ich nicht meine ganze Kraft
darauf verwenden, anstatt zu entspannen. Nun, da ich bei diesen Fragen
zu keiner zufriedenstellenden Lösung kam, rumorte das Thema in meinen
Gedanken (und im Chat mit Nicolai, der sich das gleiche schon etwas
früher als ich gefragt hat). Arme und unterentwickelte Länder bleiben
unterentwickelt und werden ärmer. Nur, wenn wir ``entwickelten'' in
unserem Eigennutz genau diese Umstände ausnutzen und geringe Löhne
zahlen (siehe unsere ``Bediensteten'') oder Land kaufen, um dann große
Villen mit den eigenen Arbeitern anstatt den einheimischen zu bauen. All
das zu verhindern ist schwierig, aber nicht unmöglich, wenn man im
Alltag bewusster darauf achtet, wo denn all das Zeug, was man so günstig
kauft, herkommt. Auch sollte man natürlich nicht wirtschaften, um
eigennützig Reichtum zu akkumulieren und auch einmal an andere denken.
All das entspricht so ziemlich der christlichen (oder allgemein
religiösen) Lehre und wir tun nach wie vor gut daran, danach zu leben.
Ok, andere nennen das dann eben unsere ``Werte''. Man vergisst das alles
aber sehr schnell und erkennt es nur wieder, wenn man mit der Nase
darauf gestoßen wird. Ich mit meiner kleinen Reise nach Neuseeland habe
ja noch eigennütziger gehandelt, hätte ich ja auch nach Afrika gehen
können, um zu helfen. Punkt. Das also als Auszug aus meinen Gedanken.
Nun sehe ich aber auch, dass die Leute hier glücklich, ja wirklich
glücklich sind. Wahrscheinlich sogar glücklicher als wir, die wir uns
sorgenfrei neue Sorgen schaffen und das dann Fortschrittlichkeit nennen.
Unsere Maßstäbe passen nicht überall, Werte aber manchmal schon eher.
Auch wenn die Leute glücklich sind, sollte man ihre Lage nicht
verschlechtern, nur um in seiner Richtung weiter zu kommen. Mit welchem
Recht zerstören wir eigentlich einen Planeten, auf dem sie noch nicht
einmal die Möglichkeit hatten genau so ``toll'' (schlimm) wie wir zu
werden. Wissen bringt ``Macht''. Naja wohl eher ``frei''. Hier auf Fiji
weiß man um den westlichen Lebensstil und steht darüber, auch wenn man
den Touristen zuliebe ein paar Spiegelbilder aufstellt und seine Sprache
zu einem einzelnen Wort ``Bulla'' (``Hallo'') verkrüppelt. Zur
Erinnerung daran wird man dann von allen Seiten damit beschmissen. Bulla
sagt der Verkäufer, an dessen Stand ich einen Bullachino bestelle,
nachdem ich mir ein Bulla-Shirt (Fiji braucht ja auch ''Hawai-Hemden'')
bei Bulla-Looks (Ok, der Laden heißt Jack's\ldots{} und ich habe mir
keines gekauft) gekauft habe. Aber zurück zum Text. Würde hier jedes
Kind Zugang zu Bildung haben, so wäre es nicht zwangsläufig glücklicher,
dafür jedoch freier zu werden, was es eben werden will. Vielleicht ist
das ein Ansatzpunkt. Auch wenn ich aus dem Wust der Gedanken, den ich
hier nicht noch weiter ausrollen möchte, den ich aber in einer OneNote
Übersicht zu systematisieren versuche, noch keine klare Linie
herausziehen kann, so habe ich doch schon eine gewisse Synthese
gewonnen. Um so mehr der einzelne voran kommt, ohne andere zurück zu
stoßen, um so mehr kommt das Ganze voran. Um so besser der Einzelne
wird, ohne anderen zu schaden, um so besser wird das Ganze. Das klingt
in meinen Ohren recht egoistisch, ist jedoch das zufriedenstellendste,
das ich bisher hervorgebracht habe. Lebe, so gut du kannst, und
verschließe deine Augen nicht vor deinen Fehlern. Sollte ich einmal zu
Reichtum kommen, so setze ich ihn weise ein, sodass er zu einem Reichtum
aller wird. Holla, Marx grüßt. Bis dann, alsbald, euer Valentin, der
sich das Ganze endlich einmal vom Herzen geschrieben hat. Ps: Ich bin
jetzt bei einem neuen Host und es ist wunderschön. Mehr dazu später.
