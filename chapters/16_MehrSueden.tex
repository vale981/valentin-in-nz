\chapdate{27.01.2017}
\chapter{Mehr S\"uden}

Ein Gefiddel ist das mit Gnu Emacs, aber man will ja nicht ewig mit
Apple TextEdit weitermachen. Nachdem ich jetzt final auf Linux
umgestiegen bin, versuche ich nur noch ``professionelle'' Linux-Hacker
Software zu benutzen. In der Tat hat der Linux-Umstieg in letzter Zeit
den größten Teil derselben aufgefressen. Aber nun habe ich mein schönes
Arch Linux Setup und kann dem Herumgetippe endlich ohne USB-Wlan Dongle
und abrupte Systemabstürze frönen.

{[}Fahrradfahren. Hechel\ldots{} Sitz viel zu niedrig, lässt sich aber
nicht auf meine außernormlichen Dimensionen einstellen.{]}

\begin{figure}[h]
  \centering
  \includegraphics[width=\textwidth]{16/Tahunanui.JPG}
  \mycap{Tahunanui Beach}
\end{figure}
Nun sitze ich - beschienen von der goldenen Abendsonne - auf einem Hügel
mit Blick auf den Tahunanui Beach und das unglaublich blaue Meer. Nach
einem sehr interessanten Film im Pseudo-Dokumentarstil musste ich mich
noch einmal abreagieren und das schöne Wetter genießen.

Auch meine letzten Tage in Collingwood waren, wie auch die Wochen davor,
sehr interessant und reich an Schönem. Ich durfte jeden Tag aufs Neue
die unglaublich fabulöse Aussicht auf die Berge genießen und, als sei
das nicht schon Freude genug, wurde auch meine Arbeit vom ausgesprochen
gutherzigen Reg Turner geschätzt.

Eines Abends fand ich einmal mehr besonderes Vergnügen daran, mit dem
ungefederten Fahrrad der Lodge über die ungeteerte Straße des
Arorere-Valleys zu touren. Ich stürzte mich also die
abenteuerlich-steile Auffahrt herunter, wurde mir unter großem Entsetzen
bewusst, dass das Betätigen der Bremsen meine halsbrecherische Tour
nicht nennenswert verlangsamt und doch kam ich dann mit sehr viel
ungewollter Mountainbike-Action auch tatsächlich heil am Ausgangstor an.
Und weiter ging es querfeldein (Staubstraße), bis ich irgendwann über
eine Brücke mit toller Aussicht bis zum Anfang des Boulder Lake
Wanderwegs radelte. Eigentlich trivial, aber wegen der schönen Aussicht
auf die Berge und das Tal trotzdem erwähnenswert. Meinen Rückweg meinte
ich durch die Wahl einer Privatstraße verkürzen zu können, hatte aber
dabei nicht einkalkuliert, wie einschüchternd die geballte Neugier von
einhundert Rindviechern sein kann.
\begin{figure}[h]
  \centering
  \includegraphics[width=\textwidth]{16/bike.JPG}
  \mycap{Radtour bei Collingwood}
\end{figure}

Tags darauf wollte ich den Milnthorpe Park besichtigen und entschloss
mich einmal mehr, anstatt des Autos das Fahrrad als Transportmittel zu
wählen.  Fleißig deichselte ich nach Collingwood, um einen Brief
abzusenden und mir auf dem Weg einen Ausguck und den alten Friedhof
anzusehen.

Kurz darauf rutschte mein Hinterrad seitlich auf der Geröllstraße (was
für eine Deichselarbeit!) weg und ich führte ein sehr akrobatisches
Ballett auf, um bis auf ein paar Schrammen an der rechten Hand
unversehrt zu überleben.

Ein paar ruhigere Minuten später durfte ich dann den Freuden von gut
angelegten Spazierwegen in schöner Natur und kostenloser Karten
hingeben, als ich den schattig-kühlen Park erreichte. Nachdem man
nicht endemische Bäume in das Brachland gepflanzt hatte, konnten auf
deren 'Ausscheidungen' und in deren Schatten auch die nativen
Pflanzenarten Fuß fassen und in den letzten 30 Jahren einen ganz
ordentlichen Wald entstehen lassen. Auf einer schönen Bank mitten im
Wald las ich dann ein wenig in Utas Neuseelandbuch, tunkte mich kurz
ins kühlende Nass und fuhr zurück nach Hause. (Wobei es der Kühlung im
verrückt-kalten Neuseeland Sommer nicht immer bedarf\ldots) )
\begin{figure}[h]
  \centering
  \includegraphics[width=\textwidth]{16/bench.JPG}
  \mycap{Ruhebank im Milnthorpe Park}
\end{figure}
\begin{figure}[h]
  \centering
  \includegraphics[width=\textwidth]{16/beach.JPG}
  \mycap{Strand am Milnthorpe Park}
\end{figure}

Die letzten Meter bergauf musste ich schieben, um den Kampf mit Kälte
und Hunger zu überstehen.  Allein der Gedanke an das Abendbrot hielt
mich auf Kurs und nach vielen Mühen wurde die Hoffnung Wahrheit,
sprich, ich aß eine doppelte Portion und war glücklich.

Am Tag vor meiner Abreise nach Nelson beschloss ich um 5 Uhr am
Nachmittag noch eine kleine Wanderung anzutreten. Ich zog auf den
Knuckle-Hill, um die Aussicht noch ein letztes Mal genießen zu können.
Dabei verkalkulierte ich mich aber gründlich, nicht nur bei der Länge
der Auffahrt, sondern auch bei der Wanderdauer, und erschien erst um
zehn Uhr abends zurück zum Abendessen! Aus einer zwei Stunden Tour wurde
eine Fünf-Stunden-Odyssee. Verwirrender Weise gab man auf dem Schild
zwar die Entfernung für Hin- und Rückweg, aber nur die halbe Zeit an!
\begin{figure}[h]
  \centering
  \includegraphics[width=\textwidth]{16/last_day.JPG}
  \mycap{Ausblick vom Berg am letzten Tag bei Reg}
\end{figure}

Gleich zum ersten Tage ein Abenteuer. Nachdem ich ausgepackt hatte,
gingen Cathy Jones, mein neuer Host, und ich einkaufen. Ich war positiv
überrascht, dass man mir sogar Pineapple-Lumps (Yummy) spendierte. Im
Verlaufe des Nachmittags ging es aber Cathys Rücken immer schlechter,
bis sie kaum noch das Auto besteigen konnte und somit gab es Takeaways
zum Abendbrot und wir fuhren zur Notaufnahme. Nachdem wir bis 12 Uhr in
der Nacht gewartet hatten (ich unter äußerst spannender Lektüre von
\url{http://www.catb.org/esr/faqs/hacker-howto.html}
wurde dann ein weiterer kollabierter Wirbel diagnostiziert, Cathy bekam
Schmerztabletten und es war an mir, den 4x4 nach Hause zu fahren.

Nach kleinen Ausflügen in die Stadt am Folgetag verbrachte ich den
Samstag mit Edith und Konsorten und mir wurde lecker Abendessen im
Lemongrass Restaurant beschert. Am Sonntag besuchte ich die Kathedrale,
denn ich muss zugeben, dass mir der Gottesdienst sehr gut zur
Gedankenstimulation gereicht und auch die Gemeinde etwas Schönes ist,
wenn man der Heimat so fern, wie ich es bin, ist. (Meine ausführlichen
Gedanken zur Religion schreibe ich aktuell nieder).
\begin{figure}[h]
  \centering
  \includegraphics[width=\textwidth]{16/rosette.JPG}
  \mycap{Rosette der Kathedrale}
\end{figure}

Montags dann wanderte ich im Zealandia-Clon\footnote{Brook Reserve} in
der Nähe von Nelson und durfte ganze 3(!) unüberbrückte Bäche
durchwaten. Eine sehr spannende Erfahrung, besonders, wenn das Wasser
dermaßen kalt ist! Abends beglückte ich Carl dann bei Alex und Pauline
mit einer kleinen Spiel-Session, nachdem er sich über die gesamte
Weihnachtszeit über einen Mangel an Zuwendung meinerseits beschwert
hatte. Finalement gab es ein wunderbares BBQ mit deutschen Würsten vom
Markt.
\begin{figure}[h]
  \centering
  \includegraphics[width=.9\textwidth]{16/lancewood.JPG}
  \mycap{Lancewood im Brook Reserve}
\end{figure}
\begin{figure}[h]
  \centering
  \includegraphics[width=.9\textwidth]{16/no_bridge.JPG}
  \mycap{Keine Br\"ucke\ldots}
\end{figure}

So viel Frischluft, wie in Neuseeland hatte ich wahrscheinlich noch nie,
denn nicht nur arbeite ich meist draußen, sondern ich wandere auch des
öfteren unter der Woche. So bestieg ich auch vorgestern einen Hügel mit
phänomenaler Aussicht über Nelson. Gestern dann war ich in
Indoor-Stimmung und so gingen Cathy und ich ins Kino, um 'Operation
Avalanche' zu sehen. (Siehe Anfang des Posts\ldots)
\begin{figure}[h]
  \centering
  \includegraphics[width=\textwidth]{16/grampview.JPG}
  \mycap{Aussicht auf den Grampians}
\end{figure}

Cathy ist ein wunderbarer Host und wir schätzen uns beide sehr. So ist
es schade, dass ich nächste Woche schon wieder weiterziehe, aber wozu
bin ich denn sonst in Neuseeland?

Ich sehe viel und erforsche die Umgebung. Dennoch ist es jedes Mal aufs
neue eine Schwierigkeit, sich umzustellen. Mittlerweile geht es aber
immer einfacher über die Bühne und ich kann auch in andere Richtungen
denken.

So schwenkte mein Interesse in letzter Zeit auf das Programmieren und
den Computer im Allgemeinen um. In den letzten zwei Wochen habe ich
meiner Meinung nach sehr tiefe Einsichten gewonnen und verstehe nun
endlich in allen Dimensionen, wozu ein Betriebssystem überhaupt da
ist.  Mal sehen, wohin und wozu mich das führt \ldots{} Bis dahin
alles Gute, Amigos!
