\chapdate{04.03.2017}
\chapter{Berge}
\begin{figure}[h]
  \centering
  \includegraphics[width=\textwidth]{19/firstview.JPG}
  \mycap{Eine erste Vorahnung der Berge}
\end{figure}

Ein nächster Tag auf Reisen. Dieses Mal in die Berge. Nach unschuldiger
Fahrt durch ein Flachland um Greymouth eröffnet sich plötzlich ein
unerwartet beeindruckender Anblick und ich drücke auf die Bremse, damit

mir die Sicht nicht so schnell wieder vom Bush verschluckt wird. Von
einer kleinen Anhöhe aus erstreckte sich einmal mehr ein Flachland, dass
alsbald jedoch in ein echtes Tal überging, umflankt von Wendelsteinen.
Es waren bei weitem nicht meine ersten Berge auf der Südinsel, doch
vielleicht die schönsten. Jeder kennt die besondere Mächtigkeit der
Berge.

\begin{figure}[h]
  \centering
  \includegraphics[width=\textwidth]{19/mountains.JPG}
  \mycap{Aussicht auf dem Temple Basin Ski Field}
\end{figure}
Zusammen mit 100 Lastwagen, die allesamt schneller vorankamen als mein
grüner Demio, hatte ich noch eine interessante und beeindruckende Fahrt
über Brücken und durch halboffene Tunnel. Unter wechselhaften Wolken,
die mir mal Regenschauer und mal Sonne bescherten, kletterte ich in
kurzen Hosen und mit gegen den Wind modifiziertem Sonnenhut (umgedreht
und die Krempen mit dem Halteband über meine Ohren gebunden) hinauf zum
Temple Basin Ski Field, bei dessen Anblick mir die Natur der
neuseeländischen Skifahrer bewusst wurde: steile Hänge, endend in
Furchen und Wasserfällen. Und auch schneebedeckte Gipfel boten sich mir
auf der anderen Seite des Tales da.
\begin{figure}[h]
  \centering
  \includegraphics[width=\textwidth]{19/snow.JPG}
  \mycap{Ein weiterer Bergblick}
\end{figure}

Zum Abend ging es weiter ins Arthurs Pass Village zur Übernachtung im
``The Sanctuary'' Hostel. Zugegeben, das Hostel war \underline{sehr}
Basic, nicht mehr als eine Tramping Hütte mit einer Küche und, Gott
sei Dank, einer Heizung, aber die Leute waren nett. Unter ihnen auch
ein deutscher Informatiker, mit dem ich im Dunkeln noch zu ein paar
beleuchteten Wasserfällen spazierte. Der Besitzer des Hostels war auch
ein lustiger Kauz, mit einem Kajakverleih in Lyttleton bei
Christchurch. Bezahlt wird im Hostel über eine Vertrauenskasse, wenn
er absent ist :). Müde ward ich und so ging es zu Bett.
