\chapdate{03.03.2017}
\chapter{Episoden}

So vieles habe ich erlebt. Um nicht gleich im Angesicht der
Niederschrift meiner Erlebnisse zusammenzubrechen, teile ich meinen Post
in kleinere, auch für den Leser angenehmere Stücke auf. So folgt auch
gleich:

\section*{Erster Teil: West nach Ost}
Nach eineinhalb sehr interessanten Wochen nahe Westport, ward es
einmal mehr an der Zeit das Heft in die Hand zu nehmen und
weiterzuziehen. Da mein Host und ich in mancherlei Hinsicht nicht ganz
auf einer Wellenlänge wahren, waren wir beide Glücklich, dass sich
zwei neue WWOOFer ankündigten und er somit einen einigermaßen
höflichen Grund gefunden hatte, mich vorzeitig fortzuschicken.

Wie tief das Problem lag, wurde mir dann erst wirklich bewusst, als
ich ein wunderbar übertriebenes Review auf meinem WWOOFing Profil
bestaunen durfte, indem zwar ein wahrer Kern, aber auch viel Falsches
und, soweit ich es Beurteilen kann, eine blanke Lüge steckt.

Trotzdem erschien mir mein Host als ehrlicher und auch umgänglicher
Mensch und es fällt mir schwer, diese, seine Reaktion zu
verstehen. Kultivierte Unzufriedenheit führt oft zu irrationalem
Verhalten und das auf beiden Seiten. Vielleicht dachte John, er müsste
die Gemeinschaft der WWOOFing Hosts vor einer so schrecklichen Gefahr,
wie ich sie in seinen Augen für den ehrlichen Arbeitgeber darstelle,
warnen. Ich für meinen Teil hatte einen deftigen Kratzer im
Lack. Meine aktuellen Hosts schätzen meine Arbeit aber sehr und siehe
da: Die Welt sieht schon viel Besser aus.

Wo ich schon einmal über vier freie Tage verfügte und es eine recht
weite Strecke bis zu meinem nächsten Ziel (Christchurch) war, lag es
nahe, die Zeit reisend (im touristischen Sinne) zu verbringen.

\begin{figure}[h]
  \centering
  \includegraphics[width=\textwidth]{18/rain.JPG}
  \mycap{Regnerische Reise}
\end{figure}
Nach anfänglichem Regenguss, verbesserte sich die Lage in Punakaiki zu
einem Grauen aber Regenfreien Regen. Mit einem deutschen Hichthiker,
den ich auf dem Wege eingesammelt hatte, spazierte ich um die
sagenumwobenen Pancacke Rocks. Ein echter Touristenfang und dazu noch
ein recht Schöner. Aber im Angesicht von geteerten Wanderwegen und
Menschenmassen, deren Autos den Parkplatz selbst an einem Regentag mit
Leichtigkeit blockieren, erkannte ich wieder einmal, welch ein Glück
ich habe, kein Tourist zu sein.

Mit dem letzten Liter Benzin und einer leuchtenden Warnanzeige
schafften wir es zuletzt noch nach Greymouth, die größte Stadt am
Westcoast und der Standort der ersten Tankstellen (Plural!
Welch eine Dekadenz!) in 100 Kilometern. Greymouth wirkt auf den
ersten Blick wie Stephen Kings Derry und auch auf den zweiten Blick
und erst recht auf den Dritten.

\begin{figure}[h]
  \centering
  \includegraphics[width=\textwidth]{18/floodwall.JPG}
  \mycap{Die Flutmauer in Greymouth}
\end{figure}
Dennoch konnte ich bei klärendem Himmel einen schönen Spatziergang an
der kilometerlangen Flutmauer, hin zum (sehr) kleinen Greymouth-Museum
unternehmen. Als Bergbaustadt kann man in Greymouth allerlei Gerät und
sogar einen (ehemaligen?)  Hafen bestaunen. Das Museum erzähl viele
kleine und interessante Geschichten, unterfüttert mit allerlei
Fotographie.

\begin{figure}[h]
  \centering
  \includegraphics[width=\textwidth]{18/computerage.JPG}
  \mycap{``Images for the Computer Age''}
\end{figure}

Da gab es einen Unternehmer, der das schnellste Dampfschiff
Neuseelands besaß. Eines Tages lief sein Schiff auf Grund und wurde
damit, um Strafzahlungen zu vermeiden, automatisch Eigentum der Stadt
Greymouth. Das Wrack wurde alsbald durch einen Mittelsmann günstig
zurück ersteigert (\ldots{} wer will schon ein Schiff kaufen, dass
selbst der ehemalige stolze Besitzer nicht mehr haben möchte \ldots{})
und der Antrieb in ein großes ehemaliges Segelschiff verpflanzt. Das
neue Schiff dampfte mit demselben Motor, aber einem vielfachen an
Frachtkapazität, immer noch mit fast derselben Geschwindigkeit seines
Vorgängers und damit weitaus schneller, als all seine Konkurrenten.

In einem Hinterzimmer fand sich eine komplette Sammlung aller National
Geographic Heften seit den 70igern und mir stach sofort eine Ausgabe
aus den späteren 80igern ins Auge. Eine recht amüsante Lektüre, aus
einer Zeit, in der Computergrafik noch ganz neu, primitiv und
unglaublich spannend war. Ich bin heute so sehr an die Wunder des
Computers gewöhnt, dass mir diese neue Perspektive eine kleine
Erleuchtung bescherte.

Stahlgraue Wellen und silberne Kieselstrände. Palmen und
Flaxbüsche. Im südlichsten subtropischen Bush Neuseelands beschloss
ich den Tag auf einer kleinen Wanderung. Der Queens Point Lookout bot
mir einen überwältigenden Ausblick auf ein Meer aus Flax, das auf
einem geradezu geometrisch abfallenden Kliff in Zerfurchte Felsen und
schließlich in den Ozean übergeht.

Ein weiter Pluspunkt für Greymouth ist das hervorragende Global
Village Hostel, das mit gemütlichen Betten, kostenlosen Kajaks und
allerhand anderen Extras besticht. Die Küche in zunehmendes Chaos
versetzend, verbrachte ich den Abend mit der Zubereitung einer frei
erfundenen Pasta-Sauce (mit echten Tomaten, nicht aus dem Glas!) und
verschätzte mich dermaßen in der Quantität, dass ich mir die Kocherei
am nächsten Tag sparen konnte. An der Qualität allerdings, gab es
nichts auszusetzen.

So kommen ein ereignisreicher Tag und ein kurzer Blogpost zu einem
Ende.


\begin{figure}[p]
  \centering
  \includegraphics[width=.9\textwidth]{18/geom_flax.JPG}
  \mycap{``Geometric Flax''}
\end{figure}
\begin{figure}[p]
  \centering
  \includegraphics[width=.9\textwidth]{18/cabbage.JPG}
  \mycap{\centering Die Bl\"utenruten eines Cabbage-Tree,
                den es am Westcoast in gro\ss{}er Zahl gibt.}
\end{figure}
