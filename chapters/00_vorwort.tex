\chapter*{Vorwort}

Hochverehrter Leser,\\
mit Freuden präsentiere ich ihnen Altes, dass neu aufgekocht wurde.
Doch das Aufkochen war es Wert, denn nunmehr können Sie die volle
literarische Pracht meiner Berichte von der anderen\footnote{\ldots{}
  manche meinen schöneren \ldots{}} Seite der Erde erheblich
cholesterin-reduziert zu genießen. Mit dem Cholesterin meine ich die
natürlich zahlreichen Rechtschreibfehler, welche dem aufmerksamen
Leser den ein oder anderen geistigen Herzinfarkt beschert haben
könnten. Der Dank für diese Überarbeitung gebührt meiner lieben
Mutter, die sich mit dem Korrigieren viel Arbeit gemacht hat und der
ich dieses Schriftstück widme.

Den Inhalt dieses Büchleins bilden Weblog Einträge, die ich während
meiner Zeit in Neuseeland (2016/17) verfasst habe. Außer der
Rechtschreibung und der Entfernung von web-spezifischen Dingen wie
z.B. Links wurde der Inhalt nicht verändert. Das original findet man
weiterhin auf \url{https://protagon.space}.


Auch für mich ist diese ``Neuauflage'' sehr interessant, da das
beschriebene so lange (in gefühlter Zeit) zurück liegt, dass ich
selbst nicht so recht glauben kann all das erlebt zu haben. Allerdings
sind natürlich ein paar Klassiker unter den Anekdoten, welche ich zu
vielerlei Anlässen mit größter Freude von mir gebe.

Die Erinnerung an meine Erlebnisse in Neuseeland haben meine
Studienjahre golden gefärbt, genau so, wie ich es erwartet habe. Die
ersten paar Kapitel zeugen von einer Zeit, die für mich sehr schwierig
war, aber der ich dennoch vieles abgewinnen kann.  Auch werden in den
folgenden Seiten viele gute Menschen erwähnt, zu denen ich nun keinen
Kontakt mehr habe. Sie sind Teil meines Lebens geworden und ihr
Eindruck verweilt in mir, auch jetzt noch, wo sich unsere Weltlinien
so weit voneinander entfernt haben. Ich sollte wirklich mal wieder
eine E-Mail schreiben.

Ich habe den Berichten ein paar Illustrationen hinzugefügt, die mir
passend erschienen. Die Auswahl ist natürlich nicht perfekt und die
Gesamtheit meiner Fotografien kann unter ... eingesehen werden.
Ich muss allerdings warnen, dass ich viele davon mit defektem
Autofokus manuell (un)scharf gestellt habe.

%%% Local Variables:
%%% mode: latex
%%% TeX-master: "../index"
%%% End:
