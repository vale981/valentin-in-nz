\chapdate{05.04.2017}
\chapter{Christchurch}

Äonen lang schrieb er nichts und ward vergessen.

\begin{figure}[h]
  \centering
  \includegraphics[width=\textwidth]{21/botanic.JPG}
  \mycap{Der Botanische Garten in Christchurch}
\end{figure}
Doch nun ist er zurück und beginnt den Post gleich hochmotiviert mit
einem Umlaut: \textbf{\textbf{Funky :)}}. In all den besagten Äonen gab
es genug Zeit, um reichlich neue Erfahrungen zu sammeln. So werde ich,
um die Geduld des Lesenden nicht zu sehr zu strapazieren, einen gröberen
Überblick geben.

Ich habe das letzte Mal vergessen zu erwähnen, dass mir während der
Fahrt von Arthurs Pass nach Christchurch ein sonderliches Phänomen der
Atmosphäre ins Auge fiel: Eine breite, dichte und tief schwarze
Wolkenfront. Ein dunkler Horizont lag vor mir und ich machte mich auf
ein erstaunliches Gewitter gefasst. Einige Kilometer später jedoch
musste ich meine Belüftung kurzzeitig auf Innenluft umschalten, da das
vermeintlich meteorologische Phänomen eines gewaltigen Waldbrandes mit
allen Manieren inklusive des Geruchs in den Port-Hills entsprang.

\begin{figure}[h]
  \centering
  \includegraphics[width=\textwidth]{21/memorial.JPG}
  \mycap{Das Earthquake Memorial}
\end{figure}
In Christchurch selbst war aber außer der Wolke nichts zu sehen und zu
bemerken. So hatte ich einen wunderbaren Tag im beeindruckend schönen
Christchurch. Ich denke, ich habe bewusst nur die schönen Dinge
wahrgenommen und dennoch kann ich nicht verstehen, das Christchurch so
wenig geschätzt wird. Am morgen hatte ich eine nette Studentin aus
meinem Hostel zur Universität gefahren und dabei haben wir uns auch
gleich für die Christchurch Gondola verabredet. Während Sie sich also in
der Universität einschrieb, spielte ich Tourist und ließ mich von den
botanischen Gärten und der Innenstadt erfreuen. Besonders der kleine
Strom ``Avon'' und das neu entstandene Earthquake-Memorial beeindruckten
mich sehr. Zum späten Nachmittag durfte ich schließlich einen
phänomenalen Ausblick von der Christchurch Gondola- und im dazughörigen
Restaurant ein Stück Käsekiuchen genießen.
\begin{figure}[h]
  \centering
  \includegraphics[width=\textwidth]{21/britflair.JPG}
  \mycap{Britsches Flair am Avon}
\end{figure}


\begin{figure}[h]
  \centering
  \includegraphics[width=\textwidth]{21/harbor.JPG}
  \mycap{Blick auf den Christchurch Harbor}
\end{figure}
Noch im Schatten meiner letzten WWOOFing Erfahrung versuchte ich, durch
Pünktlichkeit einen guten Eindruck bei meinen nächsten Hosts zu machen.
Allerdings hatte ich nicht wirklich mit der phänomenalen Verkehrslage in
Christchurch gerechnet und so kam ich eine halbe Stunde zu spät.
Anscheinend wurde das aber schon erwartet und so hatte ich einen
herzlichen und entspannten Start mit Martyn (meinem Host). Auch Sue
(dessen Gattin) war und ist herzensgut, auch wenn Sie mich mit ihrer
Direktheit zu Anfang etwas erschreckte. Da ich immer noch Probleme mit
meinem Handgelenk hatte, trug ich zum Autofahren meine Handgelenkstütze,
die dann gleich als Beeinträchtigung meiner Arbeitseffizienz gefürchtet
wurde :P. Martyn versuchte zu schlichten, aber Sue meinte, ich wäre
unfair gewesen, sie nicht über meinen Gesundheitszustand aufgeklärt zu
haben. Nach meiner ehrlichen Entschuldigung, hatte ich mir doch wirklich
nichts in dieser Hinsicht gedacht, und einer guten Arbeitsleistung am
Folgetag war das Problem dann vergessen. :)
