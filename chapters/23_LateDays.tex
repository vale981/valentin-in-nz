\chapdate{18.05.2017}
\chapter{Late Days}

Was ist nur mit dem jungen Mann. Man hört ja gar nichts mehr\ldots{}

Wie immer beginne ich auch diesmal mit einer Entschuldigung. Alles ist
beim alten und Valentin schiebt den Blog immer noch vor sich her. Der
aufwändige Produktionsprozess hat es aber auch in sich! Schreiben,
durchlesen, ausbessern und schließlich der Grammatisch-Orthografische
Korrekturgang (Ohh ja, den gibt es wirklich! Ungelogen! Ich habe ja kein
Wort über die Effektivität verloren :P.). All das verlangt einiges an
Arbeit und damit auch Überwindung.

Wie dem auch sei (another frequently used term). Nach ein paar schönen
Reisewochen mit Muddi und Falko und Noemi und ner ganzen Packung
Robertsons gab es noch zwei schöne, aber unspektakuläre Wochen in
Wellington unter dem Dach der sehr hospitablen Frau Edith. Thank You!
Ich habe mir einen recht bereichernden Vortrag über (Sonnen)Uhren bei
einem Meeting der Wellington Astronomical Society angehört, besuchte das
"Space \& Science Festival" und ward erleuchtet über Titan und die NASA
Mission zum Mars.

Wenn ich unseren roten Nachbarn auch als interessant und möglichen
Kandidaten für Kolonialisierung handle, warte ich gespannt auf die Daten
einer Europa-Sonde (Jupiter Mond, nicht Kontinent). Was passiert, wenn
wir auf einen Schlag wissen, dass Leben nicht Terra-Exklusiv ist?

\begin{verbatim}
     (_\     /_)
       ))   ((
     .-"""""""-.
 /^\/  _.   _.  \/^\
 \(   /__\ /__\   )/
  \,  \o_/_\o_/  ,/
    \    (_)    /
     `-.'==='.-'
      __) - (__
     /  `~~~`  \
    /  /     \  \
    \ :       ; /
     \|==(*)==|/
      :       :
       \  |  /
     ___)=|=(___
jgs {____/ \____}
\end{verbatim}

Weiter im Text: Es gab da natürlich die eine Sache, die mir
Kopfzerbrechen bereitete. Nachdem ich das Auto, the Mighty Demio, auf
Trademe gestellt hatte, erwartete ich, demnächst ein vielbeschäftigter,
in Anfragen ertrinkender Mann zu sein. Nichts da! Kein Mucks. Also
senkte ich den Preis und pumpte 50 Dollar in Trademe, in der Hoffnung
die fehlgeleiteten Menschen da draußen, die offensichtlich keinen guten
Wagen erkennen, wenn sie einen sehen, zum Kauf zu überreden. Immer noch
nichts. Was ist das, dass kann nicht sein! Da habe ich tatsächlich, bei
einer allzu trüben Inspektion der elektronischen Post eine (die!)
Nachfrage übersehen. Die Autorin derselben hatte zu meiner Erleichterung
auch eine Woche später Interesse und so stand der Deal. Ich pilgerte
nach Lower Hutt, ließ das Auto durchchecken und siehe da, man nahm mir
den guten, grünen Demio ohne jegliche Testfahrt oder persönliche
Inspektion ab! Edith witterte Betrug und Matt deichselte
freundlicherweise die reibungslose Übergabe mit mir!

Noch etwas zu meiner Schande: Ich Horst habe es nicht hinbekommen, mich
mit meinen Arbeitgebern zu treffen :/.

Einige Eskapaden gab es auch mit Matt, dem ich beim Einrichten einer
weiteren Webcam geholfen habe. Alles, was schiefgehen kann, ging auch
schief! Aber damit gehe ich nicht weiter ins Detail\ldots{}

Dank eines Mietwagen-Transfer-Deals hatte ich den Luxus, mit all meiner
Baggage gemächlich nach Auckland fahren zu können. Auf dem Weg machte
ich mal hier, mal da, ohne genauen Plan Halt und besuchte alte Freunde.
Zuerst Jean Hollis, deren Garten noch schöner ist, als ich mich zu
erinnern wagte, mit der ich wieder einmal Ukulele spielte und die mir
reichlich Äpfel und Fejoas bescherte. Eine wunderbare Sache und eine
merkwürdige Perspektive, wenn man jetzt, am Ende, zurück schaut. Jean
Hollis war/ist MONATE her. Welch zeitliche Dimensionen.

Weiter Nördlich, in Tauranga, hatte ich noch ein Bonbon. Ich habe Tracy
(wer erinnert sich), meine Kiwi-Mum, besucht und es nicht bereut.
Reichlich zu erzählen hatten wir und gut zu Essen auch (denn ich habe
gekocht). Was wäre bloß gewesen, hätte ich den Flieger genommen\ldots{}
Tracy macht gerade dies und jenes, erfreut sich der Diversität und hat
anscheinend ihr Ding gefunden. Housesitting, lawn-mowing, Arbeit in
einem Animal-Sanctuary (mit erstaunlich vielen Tieren) zählen dabei zu
ihren momentanen Tätigkeiten.

Gestern segelte ich dann nach einem entspannten Kaffee mit Tracy in
einer nervenaufreibenden und sehr spannenden Odyssee nach Auckland.
Zuerst Stau, dann Verkehr! Und schließlich stirbt mein Telefon. Ich
erfahrener Reisender verlasse mich natürlich exklusiv auf mein Navi und
denke nicht einmal an old-fashioned Karten\ldots{} Zum Glück war ich
gerade in der Nähe des "Museum of Transport and Technology" und die
freundlichen Menschen dort druckten mir eine Karte, mit der ich dann
eine halbe Stunde brauchte, um das Hostel (ein wunderbares) zu finden.
Und warm war es. Schweißgebadet und zitternd war ich ein paar
Kollisionen nur haarscharf entronnen, entlud mein Auto und kämpfte mich
zurück zum Flughafen, den ich dann unfreiwillig auf der Suche nach Ace
Rentals erkunde. Bei der Autovermietung war natürlich schon keine Seele
mehr und in einem Augenblick der Panik übersah ich die Schlüsselbox.
Wanderung zur Bushaltestelle + Toilette suchen + Wanderung in Auckland +
tagelang kein guter Schlaf = Guter Schlaf. Was für ein Abenteuer. Aber
mir gefällt Auckland und dabei hört man so viel schlechtes. Wenn man an
den richtigen Orten verweilt, ist es prima. Ich lebe gerade in Ponsonby,
auf dem Hügel.

Heute habe ich mir ein Paar Teslas angeschaut. Schöne Autos, auch wenn
die weiße Farbe der Sitze wohl etwas unglücklich gewählt ist. Ich bin
gespannt, wann Tesla ein erschwingliches Modell produzieren wird\ldots{}
Es war schon interessant das Auto zu sehen, nachdem man die Biografie
(eine Ode auf Musk\ldots, es wird fast schon langweilig) gelesen hat.
Danach, es regnete, ging es ins bereits erwähnte "Museum of Transport
and Technology", indem ich den Rest des Tages verbrachte. Selbst nachdem
man das Berliner Technikmuseum gesehen hat, wird es nicht langweilig.
Viel gab es zu erkunden und besonders das Multiplikationslineal hat mich
fasziniert. Auch gab es eine Ausstellung mit Neuseeländischen Startups,
unter denen sogar ein Raumfahrtunternehmen zu finden war. Ich habe
natürlich jedes Täfelchen gelesen und musste durch einen Anruf auf die
Öffnungszeiten aufmerksam gemacht werden. Morgen gehe ich wieder hin :).

PS: Interessante Dampfmaschinen gab es auch: Sogar einen, in einer
Butterfabrik benutzten, ehemaligen Schiffsmotor!

Und jetzt gehts schlafen. Bis nächste Woche in Deutschland.
