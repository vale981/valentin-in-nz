\chapdate{18.10.2016}
\chapter{Mir fallen keine Ueberschriften ein}

\begin{wrapfigure}[17]{r}{0.3\textwidth}
  \centering
  \includegraphics[width=.3\textwidth]{08/buchladen.JPG}
  \caption*{Im Buchladen in Wellington, der indirekt zu meinem Physikstudium f\"uhrte.}
\end{wrapfigure}
Mal wieder eine Meldung\ldots{}

Es waren und sind schöne Tage bei den Darwins. Wir drei Deutsche
verstehen uns prächtig und stellen uns taub, sobald jemand ein
deutsches Wort spricht. Ich habe mal wieder viel erlebt und nun die
richtige Balance gefunden. Zu meinen Erlebnissen: Mount Victoria
bestiegen (danach Nüsse gekauft! Mjamjam), Victoria Universität
besichtigt (geschockt von den Studiengebühren, aber die Bibliothek ist
umfangreich und kostenlos), gewandert (Allein, mit Hund, mit
Deutschen, mit Edith und Carl) und heute: Star Trek II im
Planetarium. Die Decke der Schusseligkeit abwerfend fällt mir ein,
dass wir gestern im Rivendell (LOTR, Stadt der Elben) Tal schwimmen
waren! Es war so kalt, dass sogar der kühle Wind angenehm
erschien. Alle LOTR Fans erblassen vor Neid!  (Ich hätte es ohne
Schild aber nicht erkannt.)

Ein erfülltes Reiseerlebnis bisher! Hinweg du Trübsal! Als Ausgleich
habe ich begonnen an einem Machine Learning Kurs teilzunehmen (Hurra,
habe ein Stipendium bekommen und spare 400\$). Die Mathematik dazu
(Lineare Algebra) ist abwechslungsreich und wunderbar neu. Eine
Matrizengleichung abzuleiten hat mich trotz Anleitung 4 Seiten Papier
gekostet. Wie der Wind steht, werde ich mich beim Studium wohl dann eher
mathematisch orientieren: Kybernetik oder Technomathematik.

Ich lebe hier an den Grenzwerten für mein Empfinden für Sauberkeit (Hund
in Wohnung, Renovierung etc.), bin aber allein dadurch schon weit über
mich hinaus gewachsen (Eigenlob, Lob, Lob, Lob, Applaus bitte!). Alkohol
werde ich aber auch weiterhin nicht anrühren, nachdem ich zwei, der
Alkoholvergiftung nicht allzu ferne Betrunkene in die Stadt gefahren
habe und am nächsten Tag vom weiteren Verlauf des Abdends des weniger
Trinkfesten der beiden erfuhr. Derselbe wachte nämlich nach einem
Filmriss auf der Straße auf, wurde von freundlichen Neuseeländern mit
ins Haus gelassen, um auf dem Sofa zu nächtigen, entfloh aber wieder, um
dann von einem Spanier ein Taxi nach Karori, wo wir wohnen, spendiert zu
bekommen. Nachdem er sich nicht mehr an die Adresse unseres bescheidenen
Heimes erinnern konnte, endete er nach Überkletterung des Zaunes ohne
die Alarmanlage auszulösen, auf dem Rasen des High Comissioners,
telefonierte so laut mit dem zweiten, schon Zuhause angekommenen,
Deutschen, dass man es bis zu uns hören konnte und fand nach erneutem,
langwierigem Beklettern des Zaunes in trunkener Tollpatschigkeit nach
Hause.

Damit gehabt euch wohl und bis zum nächsten Mal, liebe Kinder.
