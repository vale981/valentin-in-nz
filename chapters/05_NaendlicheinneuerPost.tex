\chapdate{14.09.2016}
\chapter{Na endlich ein neuer Post}

Abenteuer. Ich habe diesen Blog die letzten paar Monde (o.k. Gott sei
Dank nur einen Mond) sträflich vernachlässigt. Aber hurra! Ich habe
etwas erlebt und kann nun tatsächlich einen Blogeintrag mit Inhalt
schreiben.

Ich habe mir ein Auto (Mazda Demio) gekauft, meinen Job beendet und bin
Hals über Kopf losgeWWOOFt. Sprich: Ich habe ein paar Hosts
angeschrieben und dem ersten (und einzigen), der mir antwortete,
zugesagt.

\begin{itemize}
\tightlist
\item
  Ich bin freudig losgefahren und nach fünf Stunden in Taumarunui
  angekommen.
\item
  Ich melde mich beim WWOOFing Host und werde zum Grundstück gelotst.
\item
  Ich sehe, wie es im WWOOF Profil beschrieben war, einen Garten der in
  der nächsten Woche Objekt meiner Arbeitsbemühungen werden soll.
\end{itemize}

Voller Optimismus sehe ich das als gutes Zeichen an. Ein paar Minuten
später werden mir die Umstände meiner Unterbringung erläutert. Das flaue
Gefühl was mir schon seit geraumer Zeit im Kopfe herumspukt explodiert
im Angesicht einer unbeheizten nicht elektrifizierten Hütte, einen
halben Kilometer vom Haus des Hosts entfernt. Ich, der ich von der
Gastfreundschaft meiner lieben Tante (danke!) verwöhnt bin, halte erst
einmal mit meinen Gefühlen hinter dem Damm und sage brav ja zu allem.
Weiter bergab geht es als ich endgültig den Überblick verliere und mich
fragen muss, wie ich von ein paar Einmachdosen und einem Gaskocher leben
soll. In meiner Verzweiflung (und in Tränen aufgelöst) weder ein noch
aus wissend telefoniere ich mit Edith (meiner Tante) und ziehe in
Betracht, in einem Motel zu übernachten und am nächsten Tag den Rückweg
anzutreten. Trés Bon. Das einzige worauf ich in Hinsicht auf diese
Affäre stolz bin ist, dass ich dem Host höflich mitteilte, dass die
Situation meinen Erwartungen nicht entspräche und ich mich für die
Unannehmlichkeiten entschuldige. Der Host zeigte Verständnis und bot mir
an, mich einem Freund zu vermitteln, der mehr Erfahrung mit WWOOFING
hätte. Ich nahm das Angebot an und sah mich gleichzeitig nach einem
neuen Host um. Getrieben von einer Art Panik, fühlte ich mich doch auf
irgend eine Weise in einen Schlamassel hineingeraten, sagte ich einem
Zweiten WWOOFing Host zu. Bald darauf traf der Freund des Hosts mit
einem weiteren österreichischen WWWOOFer ein. Der versichert mir, das
sein Host und seine Unterbringung O.K. sei. Ich, ganz vertieft in meinen
Schlamassel, kam mit den beiden mit in der Erwartung, auf ähnlich
unerfreuliches zu treffen und sehe mich positiv überrascht. Wir sind in
einem alten Maori Kongresszentrum, das kürzlich den Besitzer gewechselt
hat und nun wieder auf Vordermann gebracht wird. Ich lerne eine zweiten
deutschen (!) WWOOFer kennen und darf übernachten. Am nächsten Tag will
ich nach einem arbeitsamen Vormittag zum nächsten Host aufbrechen,
entscheide mich dann aber doch zu bleiben. Nun bin ich schon den dritten
Tag hier und habe mich mit allen angefreundet. Micha, der deutsche
WWOOFer, koch gut und gerne und ich freue mich zu helfen und zu lernen
(wir speisen vorzüglich!). Paora, unser Host, ist ein guter Gastgeber
und bäckt ein vorzügliches 'Fried Bread'. Ich habe bisher vormittags im
Garten gearbeitet und nachmittags frei gehabt. Heute aber war ein
Hundswetter und wir haben eine Aufräum- und Putzaktion im Hause
gestartet. Morgen ist wieder Hundswetter und wir gehen in die heißen
Quellen! Langsam gewöhne ich mich an die Idee des WWOOFens, fühle mich
nicht mehr so hilflos und plane Ausflüge (\ldots{} ich war endlich mal
in der 'Stadt' und habe das Visitor Centre besucht). Ich habe mich aber
noch nicht entscheiden können, ob das WWOOFing leben für mich taugt.
Nichtsdestotrotz geht es wieder bergauf.

Nun muss ich eingestehen, dass ich ein Esel war:

\begin{enumerate}
\tightlist
\item
  Ich habe mir wohl nicht vorstellen können was es heißt, für sich
  allein verantwortlich zu sein.
\item
  Was hat mich geritten einen WWOOFing Host so weit im Norden (5h von
  Wellington) anzunehmen?
\item
  Warum habe ich, naiv wie ich bin, nicht weiter über die Gegebenheiten
  recherchiert?
\end{enumerate}

Der Host hatte noch keine Bewertungen. Ich habe törichter Weise
angenommen, ich könne der erste sein, der ihm eine gute Rezension
schreibt. Tatsächlich wusste er wohl nicht wirklich über das WWOOFing
Bescheid. Ich habe durch das 'We have WiFi' in der Beschreibung
angenommen, ich sei im Wohnhaus untergebracht.

Ich werde aus all dem lernen! Aber natürlich ist es empfehlenswert sich
seiner Eseleien bewusst zu werden bevor man naiv drauf los rennt!

Gehabt euch wohl!
