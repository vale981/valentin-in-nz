\chapdate{13.04.2017}
\chapter{Viel Neues}

Welch turbulente Tage. Nach einer großen Panik, die sich bei mir schon
das ganze Wochenende mit einem Unwohlsein angekündigt hat, sitze ich
jetzt auf der Rückbank unseres sehr kleinen Campervans und habe ein
wenig Ruhe, um endlich mal wieder von meinen Abenteuern zu berichten.
Die erste Nacht ist erstaunlich komfortabel überstanden und ich bin
wieder entspannt :).

Den letzten Monat habe ich noch einmal etwas Neues ausprobiert und mich
ins kalte Wasser gestürzt. Und wirklich, ich habe mich am ersten Abend
gefragt, was ich mir eigentlich gedacht habe. Harsch enttäuscht von
Dunedin und überwältigt von der Aussicht, einen Monat einmal ganz für
mich allein zu sorgen, sah ich am ersten Abend wirklich kein Licht. Ein
paar Tage später war meine "Reisekrankheit" aber auch schon wieder
kuriert. Ich hatte mich in Hogwartz, dem Hostel, in dem ich arbeitete,
eingelebt und auch mit meiner Programmierarbeit ging es voran. Zu meinen
Kollegen im Hostel konnte ich zuerst keinen Draht finden und besonders
Lukas, ein Musiker und Programmierer, gab sich sehr verschlossen. Ich
fand jedoch recht schnell heraus, dass jeder, außer einem Langzeitgast
des Hostels, dasselbe Problem hatte und knüpfte darauf hin schnell
Freundschaften mit zwei Belgierinnen. Im Allgemeinen war ich überrascht,
wie gesellig ich mich auf einmal in der Flut der neuen Menschen, die
jeden Tag über mich hereinbrach, verhielt. So unternahm ich regelmäßig
Ausflüge und immer fand sich genug Gesellschaft, um mein Auto zu füllen.
Es gibt diesen ganz bestimmten Schlag von jungen Reisenden, die sich
immer mit uns (der Belegschaft) in der kleinen und meist übersehenen
Küche neben der Großküche zusammenfanden und mit denen man immer
prächtig auskam. In besagter Küche, die ich auf Grund ihrer geringen
Popularität immer ganz für mich selbst hatte, konnte ich nach
Herzenslust Kochen, Braten und Backen, ohne mich in Rivalitäten um Töpfe
und Herdplatten zu verstricken. Auch wenn ich zuvor schon gelegentlich
ein paar Nudeln eingeweicht und Fertiggerichte nach Anleitung zubereitet
hatte, konnte ich nicht auf einen großen Erfahrungsschatz zurückblicken.
Es sei mir das Eigenlob vergeben, aber ich meine, mich sehr gut
geschlagen zu haben. Von Bolognese über gebackene Kumara bis hin zur
Lasagne hatte ich nicht unter Nahrungsmangel zu leiden. Und im
Kühlschrank stapelte sich das im ersten Einkauf erstandene Toastbrot,
denn gleich an meinem ersten Tag hatte ich wieder angefangen Brot zu
backen. In Wirklichkeit ist Brotbacken ziemlich einfach, aber sehr
lohnend und schindet deswegen nur umso mehr Eindruck. Fleißig teilte ich
mein Brot und mein Wissen, war aber der Einzige, der bis zum Ende alle
halbe Woche Brot buk.

Welch bemerkenswerte Phänomene durfte ich in unserem kleinen
(mittelgroßen) Hostel beobachten. Wenn man sieben Uhr aufstand, hatte
man das ganze Hostel für sich. Um acht konnte man den Schleier der
Trägheit noch förmlich sehen. Und in meinem sehr dunklen, aber
gemütlichen Schlafzimmer konnte ich des öfteren selbst um zehn Uhr nicht
staubsaugen, weil einige besonders bequeme Individuen immer noch in
ihren Betten ruhten. Reisende sind ein lustiges Volk, besonders die
Sorte, die mehrere Monate unterwegs ist. Hört man die Geschichten eines
solchen Weltenbummlers, so kann man sich kaum vorstellen, wie auf Reisen
auch nur ein Tag ohne neue, atemberaubende und phänomenale Eindrücke
vergehen kann. In Wirklichkeit sind die meisten Tage solcher Menschen
von an Lethargie grenzender Trägheit gekennzeichnet. Relativierend muss
ich aber gestehen, dass dieser Eindruck wahrscheinlich von Extremfällen
herrührt, die am Ende der Saison nicht mehr mit der alten Energie
umherziehen. Ich selbst hatte, da man erst um 10 Uhr zur Arbeit antrat,
alle Mühe, meinen Schlafrhythmus aufrecht zu erhalten.

Wie dem auch sei. Besonders ein bemerkenswertes Exemplar des Homo
Instrenuus wurde mir zu einem guten Freund, auch wenn ich ihren Namen
immer noch nicht kenne. Sie, eine Chinesin, entfloh dem Stress, kam für
ein Jahr nach Neuseeland und blieb dann irgendwie in Dunedin hängen.
Auch wenn ihre Ansichten zur "Partei" sehr chinesisch anmuteten, war sie
doch als biertrinkender Fußballfan so ganz und gar untypisch. Ich konnte
ihr, die sie ihren Lebtag noch kein Saxophon gesehen oder gar gehört
hatte, mit meinem Saxophonspiel eine große Freude machen. Das ging
soweit, dass ich eines Abends, nachdem wir in einem sehr schönen Café
namens "The Dog with two Tails" (sehr untypisch wollte Sie mir unbedingt
einen Drink ausgeben. Ich habe das Bier probiert, konnte aber immer noch
nichts daran finden.) waren, mitten im nächtlichen Stadtzentrum
herumjazzte. Nachdem wir eines anderen Abends zum beeindruckend
kunstvollen Choral Evensong in der wunderbar hellen neogotischen
Kathedrale gepilgert waren, erstaunte ich Sie mit meiner Ansicht, das
Reich Gottes würde niemals kommen. Nicht, dass ich der Menschheit
besonders zynisch gegenüberstehe, aber es ist so, dass sich Religion
über das Streben zum Besseren definiert. Ohne dieses Streben verlören
die Menschen recht schnell die Motivation ihr Paradies aufrecht zu
erhalten. Vielleicht ist es also besser, wenn zumindest für die jetzigen
Menschen das Reich Gottes unerreichbar bliebe. Wir leben in
interessanten Zeiten, in denen Religion teilweise an Signifikanz
verliert und wir Chancen haben, Religion ohne Autorität und Zwang in
ihren guten Seiten zu entdecken. Die Tage der Chinesin entwickeln sich
zu einem Rhythmus von besorgniserregender Abnormalität und ich hoffe,
dass Sie sich selbst etwas Gutes tut und weitergezogen ist, wenn wir
wieder in Dunedin sind.

Auch wenn man sich nach fünf Tagen Toilettenputzen entsprechend
motiviert fühlt, war die Hostel Arbeit, wenn auch keine angenehme, aber
doch eine interessante Angelegenheit. Abendliche Jammsessions,
komfortable Betten, nette Besitzer und eine gemütliche Atmosphäre
machten mir das Hostel zu einem hervorragenden Heim. Als ich dann auch
noch ein (sehr klappriges) Fahrrad leihen durfte, mit dem ich zum
Arbeiten (Programmieren) in die schmucke und sehr ruhige
Universitätsbibliothek fahren konnte, war mein Glück perfekt. Am ersten
Tage mit dem Fahrad habe ich das Fliegen und meinen Schutzengel
kennengelernt! Wie immer grub ich mir selbst ein paar Löcher und grämte
mich des öfteren, sodass mir erst, als ich die letzten Tage in einem
nicht so angenehmen Hostel verbrachte, bewusst wurde, wie gut ich es
hatte und welche Erfahrung ich gesammelt habe. Auch mein Programmierjob
brachte mir einen unermesslichen Schatz an Erfahrung, der mir jetzt
ermöglicht an einer Open-Source Planetariumssoftware mitzuwirken. Schon
auf der Banks Peninsula habe ich Grundsteine gelegt, fleißig C++
gebüffelt und mich in QT geübt.

Sehr schöne drei Wochen waren das. Wenn wir nicht gerade
\textbf{\textbf{in}} den Wolken lagen (ich wollte schon immer mal
wissen, wie das ist :P, aber man wird des Nebels schnell überdrüssig.),
hatten wir eine wunderbare Sonne und ich konnte sogar ein paar Mal vom
Anleger aus in die kühle und tropisch blaue See hüpfen. Auf unserem
Hügel sah ich Sonnenuntergänge und genoss so manchen Tee auf der
Veranda. Noch nie war ich so glücklich über mein Auto, denn ohne ist man
auf der Halbinsel verloren. James Cook hielt den ehemaligen Vulkan sogar
für eine Insel und taufte Sie, nach seinem Bortbotaniker Joseph Banks,
die "Banks Island". Auch als ich das Land mit Panoramablick auf einem
der dortigen Hügel examinierte konnte ich mir nur schwerlich vorstellen,
dass ich auf den Überresten eines mehrere tausend Meter hohen Vulkans
stehe. Bei genauerem Hinsehen kann man jedoch erkennen, das diese
zerklüfteten Hügel, die eine Halbinsel aus tentakelartigen Landzungen
bilden und im Flachland von Canterbury so fehl am Platz wirken,
vulkanischen Ursprungs sein müssen.

Des weiteren kam ich in den Genuss der Gesellschaft eines sympathischen
französischen Game-Developers. Nachdem ich Raphael, so ist sein Name,
mächtig über die Spielentwicklung ausquetscht hatte, war ich sehr
erstaunt, wie viel wissenschaftliche Forschung hinter der Computergrafik
steckt, auch wenn ich so etwas schon geahnt hatte (Verweis auf das
National Geographic Magazin in Greymouth). Dem schlossen sich viele
Diskussionen über Politik, Soziales und sogar die Kernfusion an und ich
verstehe nun, warum er am Sinn seiner Arbeit als Game Developer zweifelt
und Bienen züchten will. Welchen Dienst tut man an der Gesellschaft,
indem man den Tag vor dem Computer verbringt, um anderen zu ermöglichen,
das Gleiche zu tun und die unmittelbaren Probleme zu vergessen. Auch
wenn ich glaube, dass allein die Freude, die man sich und anderen
bringt, gewissermaßen ausgleichend wirkt. Die Dosis macht das Gift. Auch
sollte man bemerken, dass die Welt auch bei all den Problemen nicht
unbedingt vor die Hunde gehen muss. Wenn man beispielsweise Projekte wie
Wikipedia betrachtet wird klar, dass Menschen nicht für Geld sondern aus
eigenem Interesse arbeiten können. Ferner ist die Qualität dieser Arbeit
meist sogar erstaunlich hoch. So etwas wie Open Source dürfte intuitiv
gesehen eigentlich gar nicht funktionieren, in der Realität jedoch
entsteht Erstaunliches. Komplexe Systeme, wie unser Gehirn, ein
Bienenstock oder eben Kollaboration, lassen sich weder reduktionistisch
durch das Beschreiben der einzelnen Bestandteile, noch durch die
holistische Betrachtung des Ganzen verstehen. Besonders für uns
Menschen, die an bewusste Kontrolle und Planung als menschliche
Errungenschaft gewöhnt sind, ist es schwer zu akzeptieren, dass solche
stabilen und produktiven Systeme sich zwangsläufig so gefügt (adaptiert)
haben, dass sie funktionieren. Das hat doch fast etwas Poetisches, wenn
man die Logik des anthropischen Prinzips vernachlässigt.) Über dieses
und weiteres konnte man sich prima austauschen. So gut sogar, dass wir
zuletzt nicht mehr zusammen arbeiten durften, weil wir nur noch
quatschten. So mussten wir uns auf Spaziergänge und Wanderungen verlegen
:).

Viele Ausflüge wurden unternommen: ich wanderte, ich hörte Konzerte und
ich habe sogar eine Gratis-Tour zu den Albatrossen auf der Otago
Peninsula gemacht (zur Webcam, für die ich programmiert habe). Ich habe
viel gelernt. Und ich hatte viel Freude. Ich bitte die so spärliche
Berichterstattung zu verzeihen, aber es ist so schwer, Schritt zu
halten.

Jetzt geht es eine Runde reisen mit Mama, Noemi und Falko :). Auch wenn
jeder vom Wetter und der Umstellung etwas angereizt ist, wird es
bestimmt ein Spaß.

Zur Reiseberichterstattung verweise ich fauler Weise einmal an Falkos
Blog: \url{http://nz2017.trojahn.de}

Gehabt euch gut ;)
