\chapdate{22.10.2016}
\chapter{Es leben die langen Ueberschriften - Na so lang ist sie ja auch wieder nicht - Jetzt schon hahahaha reingelegt}

In dem Bemühen, meine Post-Frequenz der Frequenz der kosmischen
Hintergrundstrahlung anzupassen, verfasse ich mal noch einen kleinen
Beitrag in mein (öffentliches) 'Tagebuch'.

Vorgestern waren wir im Weta-Workshop, benannt nach der indigenen
gottesanbeterartigen Weta, die in Neuseeland einmal dieselbe Nische wie
bei uns die Mäuse besetzt hat. Dort gab es allerlei Filmrequisit und
Maskerade zu sehen. Genau das wird dort nämlich, unter anderem für LOTR
und den Hobbit, produziert. Ein Schaumstoff Stahlschwert, allerlei Äxte
und Saurons Rüstung in sicherer Schaumstoff-Spitzen Variante und
natürlich auch aus Vollmetall waren erstaunlich anzusehen, jedoch am
besten ist der Halo-Offroad-Truck. Der für einen Halo-3 Teaser
geschaffene Truck ist, auf Wunsch der Producer, voll funktionsfähig und
von Grund auf selbstgebaut. Abgefahren ist er aber nicht \ldots{} steht
immer noch dort!

Gestern dann bin ich früh aufgestanden, habe einen Deutschen zum Bus
gefahren und recht früh angefangen zu arbeiten. Dem Plan nach wollte ich
eigentlich um zwei wandern gehen, habe dann aber bis um vier getrieft,
und habe meine Wanderung um fünf angetreten. Die Lower Hutt Region bot
mir schon wieder einen neuen Natureindruck, jedoch störten die
Industrieluft und der Naheliegende Highway. Generell war es ein Tag mit
relativ wenig lichten Momenten. So etwas passiert. Ich habe gelernt
nicht zu sehr unterzutauchen.

\ldots. Brzzzzt, schwarzer Bildschirm, Akku alle.

Nächster Tag: Nichts Besonderes. Nur Mistwetter und mathematische
Beweise mit Nicolai.

Heute sollte ich eigentlich Edith, Matt und Carl gegen Zehn Uhr zum
Flughafen fahren, jedoch wandelte sich das Ganze zu einer Fahrt mit der
Fähre gegen Zwei am Nachmittag. Somit hatte ich am Vormittag reichlich
Zeit, in der aus einem wolkenfreien Himmel knallenden Sonne zu lesen.
(Das Komma ist korrekt gesetzt! Erweiterter Infinitiv mit zu!) Danach
überkam mich die große Verzweiflung über die Frage, was denn mit dem
restlichen sonnigen Samstag anzufangen sei. (Nun wir sehen: Das 'zu'
kann auch mitten im Wort stehen.) Glücklicher Weise hatte Edith die
Idee, dass ich doch den Makara-Loop-Walk machen könnte.

Im Grunde gesagt ist der Makara-Beach ein Kalenderblatt, das Gott so
sehr gefiel, dass er es in bequemer Entfernung zu Wellington entstehen
ließ. Eine der schönsten Wanderungen bisher, um es kurz zu machen.
Danach wollte ich eigentlich noch das kühlende Nass ohne Sand, denn es
war ein Kiesstrand, genießen, wurde aber nach dem Abstieg von den
Klippen aus der Bahn geworfen. Ich stolperte über ein Stück Treibholz
und geriet angesichts mehrer kleiner Schürfwunden ganz aus dem Häuschen.
Ich war schon recht erschöpft, da jeder in mir einen schnellen, straffen
Wanderer zu erkennen schien und mich vorbei ließ. Das Blut aus meinen
Schürfwunden an meiner Hose abwischend tropfte ich meinen Pullover mit
Blut aus meiner Nase voll, allerdings ohne das mitzubekommen. Erst als
ich eine Familienwandergruppe verstört hinter mir zurückließ fiel mir
auf, dass mein Gesicht nicht nur vom Sonnenbrand rot war. Naja, kein
Baden, aber ein schöner Ausflug.

Cheers!
